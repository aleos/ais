% !TEX encoding = UTF-8 Unicode
%% Лекции по Системам искусственного интеллекта. Охтилев М.Ю. (oxt@mail.ru)
\documentclass[pdftex,a4paper,12pt,%
%draft% «Черновой» режим "--- пометка overfull и underfull
]{report}
% кодировка документа
%\usepackage{ucs}
\usepackage[utf8]{inputenc}
% поддержка ссылок в pdf
\usepackage[pdftex, unicode]{hyperref}
% расстановка русских переносов, настройки для русских текстов
\usepackage[russian]{babel}
\usepackage{misccorr}
% максимальная разрежённость текста
%\tolerance=300
% поиск по русским словам в pdf
\usepackage{cmap}
% вставка иллюстраций
\usepackage[pdftex]{graphicx}
\graphicspath{{pic/}} % путь к иллюстрациям
% русские буквы в формулах
\usepackage{mathtext}
% выбор внутренней кодировки для русских букв в формулах
\usepackage[T2A]{fontenc}
% дополнительные возможности в формулах, в т.ч. правильная расстановка скобок
\usepackage{amsmath,amssymb}
% \with - логическое "и" с правильными отступами
\usepackage{cmll} % need texlive-fonts-extra
% изменённые каллиграфические буквы
%\usepackage{calrsfs}
% дополнительные шрифты с заменой \mathcal
\usepackage[mathcal]{euscript}
% дополнительные текстовые символы
\usepackage{textcomp}
% Отступ красной строки в первом абзаце
\usepackage{indentfirst}
% Более полное заполнение листа 
\usepackage{fullpage}
% дополнительные возможности для tabular 
\usepackage{array}
% \topcaption для надписей над таблицами
\usepackage{topcapt}

%\usepackage[pdftex, unicode]{hyperref}
\hypersetup{
    backref,
    bookmarks,
    bookmarksnumbered,
    colorlinks,
    pdfborder={0 0 0}, % ссылки без рамок
    pdftitle={Лекции по СИИ},
    pdfsubject={Лекции по предмету «Системы искусственного интеллекта»},
    pdfauthor={Охтилев М.Ю.},
    pdfkeywords={лекции, СИИ}
}

\usepackage{amsthm} % Теоремы, определения, замечания.
\theoremstyle{plain}
\newtheorem*{theorem}{Теорема}
\newtheorem*{conseq}{Следствие} % consequence of the theorem
\theoremstyle{definition}
\newtheorem*{defin}{Определение}
\newtheorem*{ex}{Пример}
\theoremstyle{remark}
\newtheorem*{rem}{Замечание}

\title{Системы искусственного интеллекта}
\author{Михаил Юрьевич Охтилев\thanks{e-mail: oxt@mail.ru}}
\date{\today}

\includeonly{
ch1,
ch2,
ch3,
appendix,
bibliography,
}

%\pagestyle{myheadings} 
%\headsep=.5in
%\renewcommand{\sectionmark}[1]{\markboth{}{\thesection\hspace{1em}#1}}


\begin{document}

\maketitle

\tableofcontents

\chapter{Системы искусственного интеллекта и основные принципы их построения}

\section{Этапы развития систем искусственного интеллекта}
Появление систем искусственного интеллекта (СИИ) и начало исследований
в области СИИ началось с конца 50-х годов с появлением работ Ньюэла,
Шоу.

3 этапа развития СИИ:
\begin{description}
\item[I этап:] середина 50-х--середина 60-х;
\item[II этап:] середина 60-х--середина 70-х;
\item[III этап:] с середины 70-х и до наших дней.
\end{description}

%-------------------------------------------------------------------------------
\subsection{Характеристика I этапа развития CИИ}
I этап: Разработка и создание СИИ, решающих задачи на основе
эвристических методов.

Эвристический метод "--- свойственный человеческому мышлению метод, для
которого характерно использование догадок о путях решения задач с
последующей их проверкой.

Противоположность <<алгоритму>> "--- <<эвристика>>.

Эвристические методы используются когда нельзя установить жёсткие
условия. Первоначально широкое распространение получили разработки,
моделирующие мыслительную деятельность человека.

Область применения СИИ (I этап):
\begin{itemize}
\item игры (разбиение задачи на классы, моделирование на 2-3 шага
  вперёд);
\item головоломки;
\item математические задачи познавательного характера;
\item и др.
\end{itemize}

Эти области применения характеризовались простотой и ясностью
предметной области (проблемной среды), её относительно малой
громоздкостью, возможностью прямого перебора или возможностью подбора
под модель.

%-------------------------------------------------------------------------------
\subsection{Характеристика II этапа развития CИИ}
II этап: Разработка и создание СИИ, ориентированных на применение
интегральных роботов.

При переходе на I этапе от искусственных предметных областей (ПрО) к
реальным разработчики стали сталкиваться с большими трудностями,
обусловленными необходимостью моделирования реального мира:
\begin{itemize}
\item описание знаний о внешнем мире;
\item организация хранения этих знаний;
\item эффективный поиск и доступ к знаниям;
\item проверка знаний на корректность (полнота, непротиворечивость,
  верифицируемость и пр.).
\end{itemize}

Практическое применение СИИ привело к созданию <<интегральных
роботов>>, которые привели к необходимости рассмотрения и
использования реальных ПрО.

Переход к реальным ПрО обусловлен переходом от исследований по
моделированию способов мышления человека к разработке программ,
способных решать <<человеческие задачи>>, но формальным методом
(математическая логика, смысловая логика). К этому периоду относится
разработка метода автоматического доказательства теорем на основе
метода резолюций (1969 г., Робинсон, Минский, Пайперт).

Для СИИ II этапа характерно:
\begin{itemize}
\item наличие формальной модели предметной области;
\item наличие алгоритма распознавания ситуаций, изображений, сцен;
\item алгоритм принятия решений;
\item алгоритмы планирования работы интегральных роботов;
\item алгоритм оценки качества выполненных/спланированных работ;
\item и др.
\end{itemize}

Появление первых экспериментальных образцов роботов, решающих свои
задачи в комплексе показало необходимость решения фундаментальных
проблем, связанных со следующими задачами:
\begin{itemize}
\item представление знаний;
\item зрительное восприятие (распознавание образов);
\item построение сложных планов поведения в динамических средах;
\item общение с роботами на естественном языке;
\item и др.
\end{itemize}

Всё это привело к III этапу развития СИИ.

%-------------------------------------------------------------------------------
\subsection{Характеристика III этапа развития СИИ}
III этап: Разработка и создание СИИ, характерной чертой которых
явился переход от создания автономно функционирующих систем,
самостоятельно решающих задачу в реальной среде, к созданию
человеко-машинных систем, совмещающих достоинства интеллекта человека
и возможности ЭВМ для достижения общей цели решения задачи,
поставленной перед интегральной человеко-машинной решающей системой.

Причины возникновения человеко-машинных систем:
\begin{enumerate}
\item к этому времени выяснилось, что даже простые задачи
  функционирования интегрального робота в реальных ПрО не могут быть
  решены методами, разработанными для роботов в экспериментальных ПрО;
\item Стало ясно, что сочетание дополняющих друг друга способностей
  человека и возможностей ЭВМ позволяет обойти нерешаемые задачи
  (острые углы) путём перекладывания на человека тех функций, которые
  пока недоступны для ЭВМ.
\end{enumerate} 

\begin{table}[ht]
  \centering
  \begin{tabular}{p{.35\linewidth}|@{\qquad}p{.35\linewidth}}
    \textbf{Способности человека:}
    \begin{itemize}
    \item интуиция;
    \item эвристика.
    \end{itemize}
   & 
    \textbf{Способности ЭВМ:}
     \begin{itemize}
    \item скорость;
    \item хранение данных;
    \item надёжность.
    \end{itemize}
  
    \end{tabular}
\end{table}

СИИ "--- система, в которой:
\begin{itemize}
\item развиваются возможности ЭВМ в направлении обеспечения
  совместного с пользователем решения задач;
\item упрощаются процессы общения человека и ЭВМ в ходе решения задач;
\item постоянно расширяется доля компьютеров в совместной с человеком
  деятельности по решению задач;
\item значительное внимание уделяется повышению способности ЭВМ к
  самостоятельному решению (в автоматическом режиме) трудно решаемых
  задач.
\end{itemize}

%===============================================================================
\section{Новая информационная система обработки информации}

\subsection{Основные черты традиционной (старой) обработки данных}
\begin{itemize}
\item Наличие следующих этапов обработки информации (рис.~\ref{fig:traditional_data_processing}):
  \begin{enumerate}
  \item Специалист (эксперт) "--- конечный пользователь.
  \item Системный аналитик.
  \item Программист.
  \item Вычислительная система.
  \end{enumerate}

  \begin{figure}[ht]
    \centering
      %\fboxsep=0pt \fboxrule=1pt
      \begin{picture}(450,160)
        %% \framebox — бокс заданного размера
        %% \dashbox — бокс заданного размера с пунктирной границей
        %% \parbox — абзац текста заданного размера
        %% \makebox — строка текста заданной длины
        \thicklines % «толстые» линии
        %\put(0,0){\dashbox{5}(450,160){}}
        % Блоки с надписями (этапы)
        \put(0,75){\framebox(90,50){\parbox[c][50pt]{90pt}{\centering
              Специалист\par
              Эксперт\par
              (КП)\footnotemark}}}
        \put(120,75){\framebox(90,50){\parbox[c][50pt]{90pt}{\centering
              Системный
              аналитик}}}
        \put(240,75){\framebox(90,50){\parbox[c][50pt]{90pt}{\centering
              Программист}}}
        \put(360,75){\framebox(90,50){\parbox[c][50pt]{90pt}{\centering
              Вычисли\-тельная
              система}}}
        % Стрелки между блоками в обе стороны
        \put(90,110){\vector(1,0){30}}
        \put(120,90){\vector(-1,0){30}}
        \put(210,110){\vector(1,0){30}}
        \put(240,90){\vector(-1,0){30}}
        \put(330,110){\vector(1,0){30}}
        \put(360,90){\vector(-1,0){30}}
        % Кривая стрелка сверху с надписью
        \qbezier(170,125.5)(225,155)(280,125.5)
        \put(280,125.6){\vector(2,-1){0}}
        \put(225,145){\makebox[0pt]{\small{Алгоритм, входные данные}}}
        % Скруглённый прямоугольник снизу с надписью
        \put(225,20){\oval(350,40)}
        \put(50,16){\makebox[350pt]{
            Результаты решения задачи на каждом этапе решения}}
        % Стрелки от нижнего скруглённого прямоугольника
        \put(105,40){\vector(0,1){50}}
        \put(225,40){\vector(0,1){50}}
        \put(345,40){\vector(0,1){50}}
      \end{picture}

    \caption{Этапы обработки информации при традиционной (старой) обработке данных}
    \label{fig:traditional_data_processing}
  \end{figure}

  \footnotetext{Конечный пользователь}

\item Большая трудоёмкость решения конкретной задачи обработки
  информации, связанная с имеющимися место на каждом этапе:
  \begin{itemize}
    \item ошибками;
    \item неточностями;
    \item нерациональными решениями;
    \item необходимостью внесения изменений в схемы решения, методы
      решения, постановку задачи, программу.
  \end{itemize}
\item Сложность сопровождения разработанного программного обеспечения
  (ПО).
\item ПО в традиционной технологии основывается на формальной
  (математической) модели задач и представления данных, в то время как
  каждая конкретная предметная область основывается на системе,
  содержательных понятий, которыми оперирует пользователь при решении
  задач. 

  Это значит, что в такой схеме решения система понятий ПрО и
  формальной модели не совпадают.
\item При формулировке задачи конечный пользователь должен перевести
  постановку задачи, выраженной в системе ПрО в постановку задачи
  формальной модели (рис.~\ref{fig:interpretation}), при получении
  результата "--- наоборот (рис.~\ref{fig:opposite_interpretation}).
\end{itemize}

\begin{figure}[ht]
  \centering
  \begin{picture}(450,75)
    \thicklines % «толстые» линии
    % \put(0,0){\dashbox{5}(450,75){}}
    
    \put(55,50){\oval(110,50)}
    \put(0,47){\parbox[c][50pt]{110pt}{\centering
        Формулировка
        задачи}}
    
    \put(132,46){\bfseries \Large :}

    \put(160,25){\framebox(110,50){\parbox[c][50pt]{110pt}{\centering
          Содержательная
          постановка
          задачи}}}

    \put(270,50){\vector(1,0){70}}
    \put(300,34){\bfseries \large *}

    \put(340,25){\framebox(110,50){\parbox[c][50pt]{110pt}{\centering
          Формальная
          постановка
          задачи}}}

    \put(10,0){{\large *} "--- интерпретация}
  \end{picture}
  \caption{Процесс интерпретации постановки задачи}
  \label{fig:interpretation}
\end{figure}

\begin{figure}[ht]
  \centering
  \begin{picture}(450,75)
    \thicklines % «толстые» линии
    % \put(0,0){\dashbox{5}(450,75){}}
    
    \put(55,50){\oval(110,50)}
    \put(0,47){\parbox[c][50pt]{110pt}{\centering
        Оценка\\
        результатов}}
    
    \put(132,46){\bfseries \Large :}

    \put(160,25){\framebox(110,50){\parbox[c][50pt]{110pt}{\centering
          Формальные результаты}}}

    \put(270,50){\vector(1,0){70}}
    \put(297,34){\bfseries \large **}

    \put(340,25){\framebox(110,50){\parbox[c][50pt]{110pt}{\centering
          Содержательные
          результаты}}}

    \put(10,0){{\large **} "--- обратная интерпретация}
  \end{picture}
  \caption{Процесс интерпретации полученного результата решения задачи}
  \label{fig:opposite_interpretation}
\end{figure}

%-------------------------------------------------------------------------------
\subsection{Основная идея новой технологии обработки данных}
Система понятий конкретной предметной области рассматривается как
исходная информация для решения прикладных задач, при этом
обеспечивается автоматическая интерпретация системы понятий формальной
модели.

Для новой технологии характерна следующая цепочка обработки данных (рис.~\ref{fig:new_data_processing}):
\begin{enumerate}
\item Специалист (эксперт) "--- конечный пользователь;
\item Содержательная постановка задачи;
\item Вычислительная система.
\end{enumerate}

\begin{figure}[ht]
  \centering
  \begin{picture}(400,85)
    \thicklines % «толстые» линии
    %\put(0,0){\dashbox{1}(400,85){}}
    
    \put(0,35){\framebox(100,50){\parbox[c][50pt]{100pt}{\centering
          Специалист\\
          Эксперт\\
          (КП)}}}

    \put(100,65){\parbox[c][50pt]{200pt}{\centering\small
        Содержательная постановка задачи}}
    
    \put(100,60){\vector(1,0){200}}

    \put(300,35){\framebox(100,50){\parbox[c][50pt]{100pt}{\centering
          Вычислительная
          система
          (ВС)}}}

    \qbezier(83,33)(200,-7)(320,34.5)
    \put(80,34.85){\vector(-2,1){0}}

    \put(200,2){\makebox[0pt]{\small Содержательные результаты}}

  \end{picture}
  \caption{Цепочка обработки данных в новой технологии}
  \label{fig:new_data_processing}
\end{figure}

В новой технологии обработки информации вычислительной системе должна
задаваться только постановка задачи в виде описания требуемого
результата и условий его получения, в то время как последовательность
операций, посредством которых она решается, определяется решающей
задачу системой.

Вычислительная система в составе человеко-машинной системы
превращается из пассивного звена в активную целенаправленную систему,
целью которой является получение требуемого в постановке задачи
результата или знания.

Получение этого знания достигается:
\begin{itemize}
\item автоматическим синтезом (генерацией) вычислительной системы;
\item выполнением оптимальной в определённом смысле последовательности
  вычислительных, логических и поисковых операций над имеющимися знаниями.
\end{itemize}

%-------------------------------------------------------------------------------
\subsection{Структура ВС в новой технологии обработки информации}
Основные структурные элементы вычислительной системы (ВС) (см. рис.~\ref{fig:new_computing_system}):
\begin{enumerate}
\item Модель языка пользователя.
\item Модель представления знаний.
\item База знаний.
\item Интеллектуальный интерфейс:
  \begin{itemize}
  \item система общения;
  \item решатель.
  \end{itemize}
\item Исполнительная система.
\end{enumerate}

\begin{figure}[htb]
  \centering
  \begin{picture}(450,200)
    \thicklines % «толстые» линии
    %\put(0,0){\dashbox{1}(450,200){}}
    
    \put(95,100){\dashbox{7}(300,100){}}
    \put(105,140){\framebox(100,50){\parbox[c][50pt]{100pt}{\centering
          Система\\
          общения}}}
    \put(205,165){\vector(1,0){80}}
    \put(205,165){\vector(-1,0){0}}
    \put(285,140){\framebox(100,50){\parbox[c][50pt]{100pt}{\centering
          Решатель}}}
    \put(385,165){\vector(1,0){45}}
    \put(385,165){\vector(-1,0){0}}
    \put(245,105){\makebox[0pt]{Интеллектуальный интерфейс}}

    \put(430,0){\framebox(20,200){\parbox[c][200pt]{20pt}{\centering\rotatebox{90}{
            Исполнительная система}}}}

    \put(0,150){\parbox[c]{75pt}{\centering
        Модель\\
        языка\\
        пользователя}}
    \put(37.5,125){\vector(0,-1){50}}
    \put(37.5,125){\vector(0,1){0}}
    \put(0,45){\parbox[c]{75pt}{\centering
        Модель\\
        представления\\
        знаний}}
    
   %  {
   %   \thinlines
   %   % Верхний ромб для рисования верхней поверхности
   %   \put(150,60){\line(4,1){95}}
   %   \put(340,60){\line(-4,1){95}}
   %   \put(150,60){\line(4,-1){95}}
   %   \put(340,60){\line(-4,-1){95}}
   %   % Половина нижнего ромба для рисования нижней поверхности
   %   \put(150,30){\line(4,-1){95}}
   %   \put(340,30){\line(-4,-1){95}}
   % }
   % % Вспомогательные точки верхней поверхности:
   % \put(197.5,71.875){\circle*{5}} % левая точка верхней дуги
   % \put(292.5,71.875){\circle*{5}} % правая точка верхней дуги

   % \put(197.5,48.175){\circle*{5}} % левая точка нижней дуги
   % \put(292.5,48.175){\circle*{5}} % правая точка верхней дуги

   % \put(245,83.75){\circle*{5}} % верхний угол ромба
   % \put(245,36.25){\circle*{5}} % нижний угол ромба

   % % Вспомогательные точки нижней поверхности
   % \put(197.5,18.175){\circle*{5}} % левая точка нижней дуги
   % \put(292.5,18.175){\circle*{5}} % правая точка нижней дуги
   
   % \put(173.75,24){\circle*{2}} % опорная левой дуги
   % \put(316.25,24){\circle*{2}} % опорная правой дуги

   % Верхняя поверхность:
   \qbezier(197.5,71.875)(245,83.75)(292.5,71.875) % верхняя дуга
   \qbezier(197.5,48.175)(245,36.25)(292.5,48.175) % нижняя дуга
   \qbezier(197.5,71.875)(150,60)(197.5,48.175) % левая дуга
   \qbezier(292.5,71.875)(340,60)(292.5,48.175) % правая дуга

   % Боковые грани
   \put(173.75,30){\line(0,1){30}}
   \put(316.25,30){\line(0,1){30}}
 
   % Нижняя поверхность:
   \qbezier(197.5,18.175)(245,6.25)(292.5,18.175) % нижняя дуга
   \qbezier(197.5,18.175)(173.75,24)(173.75,30) % нижняя левая дуга
   \qbezier(292.5,18.175)(316.25,24)(316.25,30) % нижняя правая дуга

   % Подпись «БЗ»
   \put(245,20){\makebox[0pt]{\Large БЗ}}
 
   % Левая верхняя стрелка от БЗ
   \put(197.5,71.875){\line(-3,1){50}}
   \put(147.5,88.541667){\vector(0,1){51,45833}}
   % Правая верхняя стрелка от БЗ
   \put(292.5,71.875){\line(3,1){50}}
   \put(342.5,88.541667){\vector(0,1){51,45833}}
   % Левая стрелка
   \put(85,45){\vector(1,0){88.75}}
   % Правая стрелка
   \put(316.25,45){\vector(1,0){113}}
   \put(316.25,45){\vector(-1,0){0}}
      
  \end{picture}
  \caption{Цепочка обработки данных в новой технологии}
  \label{fig:new_computing_system}
\end{figure}

Система программных и аппаратных средств, обеспечивающих для конечного
пользователя (КП), не имеющего специальной подготовки в области
вычислительной техники использования ЭВМ для решения задач,
возникающих в сфере профессиональной деятельности либо без
посредников-программистов либо с незначительной их помощью, называется
\emph{интеллектуальным интерфейсом}. 

Процесс же внедрения средств интеллектуального интерфейса в
вычислительную технику называется интеллектуализацией ЭВМ.  Под
термином <<\emph{интеллектуализация}>> понимается (подчёркивается) то,
что с одной стороны имеется нетривиальность функций, выполняемых этим
интеллектуальным интерфейсом, и с другой стороны то, что выполнение
этих функций до последнего времени является прерогативой человека.

\emph{Система общения} "--- совокупность средств, осуществляющих трансляцию с
языка пользователя (конечного пользователя) на язык представления
знаний в базе знаний (БЗ) и включающую в себя средства трансляции и
средства, обеспечивающие понимание.

\emph{Решатель} "--- совокупность средств, обеспечивающих в диалоге с
пользователем автоматический синтез программы решения задач,
т.е. анализ условий задачи, выделение подзадач, имеющих стандартные
решения, и объединение этих подзадач в единую целостную
программу. Функционирование системы, имеющей в своем составе решатель,
может осуществляться в 2-х режимах: в режиме компиляции и режиме
интерпретации.

\emph{Исполнительная система} представляет собой совокупность средств,
выполняющих (исполняющих) синтезированную программу вычислений,
сформированную с позиции эффективного решения задач и имеет проблемную
ориентацию.

\emph{База знаний} занимает центральное положение по отношению к остальным
компонентам вычислительной системы. Через БЗ осуществляется интеграция
всех средств вычислительной системы, участвующих в решении конкретных
задач.

Базу знаний для вычислительных систем, ориентированных на
использование новой технологии обработки данных целесообразно строить
как 2-уровневую структуру, включающую 2 компонента:
\begin{enumerate}
\item концептуальную БЗ (верхний концептуальный уровень);
\item базы данных (БД) (нижний информационный уровень).
\end{enumerate}
Такое деление позволяет обеспечить эффективное представление
обобщенных знаний и метазнаний на верхнем концептуальном уровне и
конкретной информации (данных) на нижнем уровне.

Использование баз данных в качестве отдельного компонента баз знаний
позволяет использовать современные СУБД для описания взаимодействий
(взаимоотношений) между объектами проблемной среды, причем на уровне
конкретных фактов.

\emph{Язык представления знаний} "--- это конкретный способ описания знаний в
проблемной среде, задаваемый синтаксисом описания и правилами
соотнесения языковых выражений с конкретными объектами проблемной
области или интерпретацией.

% ===============================================================================

% ===============================================================================
% ===============================================================================

%%% Local Variables: 
%%% mode: latex
%%% TeX-master: "lections"
%%% End: 


\chapter{Логические основы построения систем
искусственного интеллекта}

\section{Формальные теории (логические исчисления)}

Определение формальной теории (ФТ): любая формальная теория оперирует
с 2 объектами:
\begin{enumerate}
\item языком (определенным множеством высказываний, имеющих смысл с
точки зрения этой теории);
\item совокупностью теорем (подмножеством языка и состоящих из
высказываний, истинностных данной теории).
\end{enumerate}
Формальная система:
$$ FS = \left<A,A_1,R\right>\,,
$$
где $A$ "--- алфавит, $A_1$ "--- аксиомы, $R$ "--- правила вывода.

Формальная теория (исчисление) определяется следующим кортежем:
$$ FT=\left<U,L,S,R\right>\,,
$$
где $U$ "--- алфавит формальной теории, $L$ "--- язык формальной теории,
$S$ "--- аксиомы формальной теории, $R$ "--- правила вывода.


\begin{rem}[отличие формальной теории от формальной системы] Аксиомы и
теоремы формальной системы (ФС) трактуются как некоторые формулы или
предложения или как правильно построенные формулы (ППФ) без придания
им какого-либо смысла. Это означает, что в формальных системах
правильно построенные формулы сводятся к рассмотрению их структур или
синтаксических свойств соответствующих предложений.
\end{rem}

Рассмотрим более подробно элементы формальной теории:
\begin{itemize}
\item[$U$ "---] алфавит формальной теории:
  \begin{itemize}
  \item символы предметных констант $\{a,b,c,\ldots\}$;
  \item символы предметных переменных
$\{x,y,z,\ldots,x_1,y_1,z_1,\ldots\}$;
  \item символы функциональных констант
$\{f_1^{n1},f_2^{n2},f_3^{n3},\ldots\}$;
  \item символы предикатных констант $\{A,B,C,\ldots,X,Y,Z,\ldots\}$;
  \item символы предикатных переменных $\{\mathcal{A,B,C,\ldots}\}$;
  \item символы логических связок и кванторов
$\{\lor,\&,\lnot,\supset,\exists,\forall,\ldots\}$;
  \item вспомогательные символы $\{\;,\;,\;.\;,\;)\;,\;(\;\}$.
  \end{itemize}
\item[$L$ "---] язык формальной теории; определяется индивидуально, с
помощью различных конструктивных процедур и, как правило, рекурсивно.
\item[$S$ "---] аксиомы, $S\subseteq L$.

  Отличие формальной теории от формальных систем: каждой правильно
построенной формуле (ППФ) может быть сопоставлено некоторое
истинностное значение в каждой конкретной интерпретации.

  В формальной теории различают следующие виды ППФ:
  \begin{enumerate}
  \item \emph{Общезначимые} формулы, или тавтологии, "--- являются
тождественно истинностными во всех интерпретациях.
  \item \emph{Выполнимые} формулы "--- принимают истину или ложь в
зависимости от интерпретации.
  \item \emph{Противоречивые} "--- ложны во всех интерпретациях.
  \end{enumerate}

\item[$R$ "---] правила вывода.

  Обозначают $F_1,F_2,\ldots,F_n\vdash G$ или $\frac
{F_1,F_2,\ldots,F_n} G$, где $F_1,F_2,\ldots,F_n,G\in L$
  
  Выводом некоторой формулы $B$ из $A_1,A_2,\ldots,A_m$ называется
такая последовательность формул $F_1,F_2,\ldots,F_n$, что $F_m=B$, а
каждая $F_i$ есть:
  \begin{enumerate}
  \item Аксиома.
  \item Форма посылок ($F_i\in \{A_1,A_2,\ldots,A_m\}$).
  \item Выводима ($F_i\vdash F_1,F_2,\ldots,F_{i-1}$).
  \end{enumerate}

  Этот факт записывается $A_1,A_2,\ldots,A_m\vdash B$. В этом случае
$A_1,A_2,\ldots,A_m$ "--- посылки (гипотезы) вывода, $B$ "--- теорема
(вывод).

  Доказательством формулы $B$ в формальной теории называется вывод этой
формулы из пустого множества формул. В этом случае посылками являются
только аксиомы. В таком случае говорят, что формула $B$ выводима
(доказуема) в формальной теории.
\end{itemize}

%===============================================================================
\section{Исчисление высказываний}

Определение формальной теории исчисления высказываний:
$FT_{ИВ}=\left<U,L,S,R\right>$,
\begin{description}
\item[$U$]:
  \begin{itemize}
  \item переменные высказываний (пропозициональные символы):
$A,B,C.\ldots$;
  \item символы логических связок: $\lor,\&,\supset,\lnot$;
  \item дополнительные символы;
  \item иные способы задания $U$ нет.
  \end{itemize}
\item[$L$]:
  \begin{itemize}
  \item высказывательная переменная является ППФ;
  \item если $\mathcal A$ и $\mathcal B$ "--- ППФ, то такими формулами
являются также $\mathcal{A\lor B}$, $\mathcal{A\;\&\;B}$,
$\mathcal{\lnot B}$, \ldots
  \item других способов задания ППФ нет.
  \end{itemize}

  \begin{rem} Формулы, состоящие из единственного символа или
высказывательной переменной, называются \emph{атомарными}
(элементарными).
  \end{rem}
  
\item[$S$]:
  \begin{description}
  \item[$S_{I}$]:
    \begin{enumerate}
    \item $A\supset \left(B\supset A\right)$;
    \item $\left(A\supset
B\right)\supset\Bigl(\bigl(A\supset\left(B\supset
C\right)\bigr)\supset\left(A\supset C\right)\Bigr)$;
    \item $(A\with B)\supset A$;
    \item $(A\with B)\supset B$;
    \item $A\supset\bigl(B\supset (A\with B)\bigr)$;
    \item $A\supset(A \lor B)$;
    \item $B\supset(A \lor B)$;
    \item $\left(A\supset C\right)\supset\Bigl(\left(B\supset
C\right)\supset\bigl(\left(A\lor B\right)\supset C\bigr)\Bigr)$;
    \item $\left(A\supset B\right)\supset\bigl(\left(A\supset\lnot
B\right)\supset\lnot A\bigr)$;
    \item $\lnot\lnot A\supset A$.
    \end{enumerate}
  \item[$S_{II}$]:
    \begin{enumerate}
    \item $A\supset(B\supset A)$;
    \item $\bigl(A\supset\left(B\supset
C\right)\bigr)\supset\bigl(\left(A\supset
B\right)\supset\left(A\supset C\right)\bigr)$;
    \item $\left(\lnot A\supset\lnot B\right)\supset \bigl(\left(\lnot
A\supset B\right)\supset A\bigr)$;
    \item[*.] $A\lor B=\lnot A\supset B$;
    \item[*.] $A\with B=\lnot(A\supset\lnot B)$.
    \end{enumerate}
  \end{description}
  \begin{rem} Системы аксиом $S_1$ и $S_2$ равнозначны (эквивалентны,
т.\,е. порождают одно и то же множество ППФ. Выбор той или иной
системы определяется конкретными задачами. Для $S_1$ выводы будут
короче, но она более богатая; $S_2$ "--- более обозримая, более
компактная.
  \end{rem}
\item[$R$]:
  \begin{enumerate}
  \item \emph{Правило подстановки}. Если $\mathcal A$ "--- некоторая
выводимая формула $\mathcal A (A)$, то выводима и форма, получающаяся
из первой заменой всех вхождений $A$ на формулу $\mathcal B$:
$\mathcal{A(B)}$. Обозначение:
    $$ \frac{\mathcal A (A)}{\mathcal A (\mathcal B)}\quad или\quad
    \mathcal A (A) \vdash \mathcal A (\mathcal B)\,.
    $$
  \item \emph{Modus Ponens} (\emph{MP}) "--- правило заключения. Если
$\mathcal A$ и $\mathcal{A\supset B}$ "--- выводимые формулы, то
выводима и формула $B$. Обозначение:
    $$ \mathcal{\frac{A,\;A\supset B}B}\quad или\quad
    \mathcal{A,\;A\supset B\vdash B}\,.
    $$
  \item \emph{Теорема дедукции.} Пусть некоторая $\Gamma$ "---
множество формул, $A$, $B$ "--- тоже формулы, тогда если
    \begin{gather*} \Gamma,\mathcal A\vdash \mathcal B \Rightarrow
\Gamma \vdash \mathcal A \supset \mathcal B\,,\\ \frac{\mathcal
A}{\mathcal{A\supset B}}\,,\quad \text{где}\ \mathcal B\ \text{---
любая ППФ}\,.
    \end{gather*}
  \end{enumerate}
  \begin{rem} Система аксиом $S$ и правила $R$ (вместе с $U$ и $L$)
полностью определяют множество всех ППФ исчисления высказываний.
  \end{rem}

  \begin{ex}[вывода в исчислении высказываний (ИВ)] Покажем
выводимость формулы, воспользовавшись $S_{II}$:
    \begin{enumerate}
    \item $\frac{II.2\;(B)}{II.2\;(A\supset A)}$,
$\frac{II.2\;(C)}{II.2\;(A)}$:

      $\vdash\Bigl(A\supset\bigl(\left(A\supset A\right)\supset
A\bigr)\Bigr)\supset\Bigl(\bigl(A\supset\left(A\supset
A\right)\bigr)\supset\left(A\supset A\right)\Bigr)$.
    \item $\frac{II.1\;(B)}{II.1\;(A\supset A)}$:\quad $\vdash
A\supset\bigl(\left(A\supset A\right)\supset A\bigr)$.
    \item MP 1, 2:\quad $\vdash\bigl(A\supset\left(A\supset
A\right)\bigr)\supset\left(A\supset A\right)$.
    \item $\frac{II.1\;(B)}{II.1\;(A)}$:\quad $\vdash
A\supset(A\supset A)$.
    \item MP 3, 4:\quad $\vdash(A\supset A)$.
    \end{enumerate}
  \end{ex}
\end{description}

%-------------------------------------------------------------------------------
\subsection{Логические функции. Определение логической функции}

Рассмотрим некоторое множество $B_2\{0,1\}$, а также различные
операции, которые определены на этом множестве. Таким образом можно
сформулировать алгебру логических функций.

Алгебра, образованная множеством $B_2$ вместе со всеми возможными
операциями на нём называется \emph{алгеброй логики}.

Функцией алгебры логики от $n$ переменных называется $n$-арная
операция на множестве $B_2$.
$$ \left\{
\begin{aligned} f^{(n)}(x_1,x_2,\ldots,x_n)\in\{0,1\},\\ \forall
x_1,x_2,\ldots,x_n \in\{0,1\}.
\end{aligned} \right.
$$
$$ f^{(n)}\colon \underbrace{B_2 \times B_2 \times \ldots \times
  B_2}_{\text{n раз}}\rightarrow B_2\quad\Rightarrow\quad
f^{(n)}\colon B_2^n\rightarrow B_2 \,.
$$

Все функции $f^{(n)}$ объединяются множеством $P_2=\{f^{(n)}\}\,$.

Алгебра, образованная $k$-элементным множеством $B_k\,,\;k=1,2,\ldots$
вместе со всеми операциями на нём, называется \emph{алгеброй
$k$-значной логики}. $n$-арные операции на $k$-элементном множестве
$B_k$, называются $k$-значными логическими функциями. Множество всех
$k$-значных логических функций $\{f_k^{(n)}\}=P_k\,$.

%-------------------------------------------------------------------------------
\subsubsection{Логические функции одной переменной}

$n=1$.

\begin{center}
  \begin{tabular}{c || c | c | c | c |} $x$ & $\varphi_0$ &
$\varphi_1$ & $\varphi_2$ & $\varphi_3$\\ \hline\hline 0 & 0 & 0 & 1 &
1\\ \hline 1 & 0 & 1 & 0 & 1\\ \hline
  \end{tabular}
\end{center}

$\varphi_0(x)=0\,$,\quad $\varphi_1(x)=x\,$,\quad
$\varphi_2(x)=\overline\varphi_2\,$,\quad $\varphi_3(x)=1\,$.

\subsubsection{Логические функции 2-х переменных} $n=2$

\begin{center} \tabcolsep=5dd
  \begin{tabular}{c|c||c|c|c|c|c|c|c|c|c|c|c|c|c|c|c|c|}
$x_1$&$x_2$&$\varphi_0$&$\varphi_1$&$\varphi_2$&$\varphi_3$&$\varphi_4$&$\varphi_5$&$\varphi_6$&$\varphi_7$&$\varphi_8$&$\varphi_9$&$\varphi_{10}$&$\varphi_{11}$&$\varphi_{12}$&$\varphi_{13}$&$\varphi_{14}$&$\varphi_{15}$\\
\hline\hline 0&0&0&0&0&0&0&0&0&0&1&1&1&1&1&1&1&1\\ \hline
0&1&0&0&0&0&1&1&1&1&0&0&0&0&1&1&1&1\\ \hline
1&0&0&0&1&1&0&0&1&1&0&0&1&1&0&0&1&1\\ \hline
1&1&0&1&0&1&0&1&0&1&0&1&0&1&0&1&0&1\\ \hline
  \end{tabular}
\end{center}

\begin{center}
  \begin{align*} \varphi_0(x_1,x_2)&=0\,,&
\varphi_1(x_1,x_2)&=x_1\with x_2\,,&
\varphi_2(x_1,x_2)&=\overline{x_1\to x_2}\,,\\
\varphi_3(x_1,x_2)&=x_1\,,&
\varphi_4(x_1,x_2)&=\overline{x_1\leftarrow x_2}\,,&
\varphi_5(x_1,x_2)&=x_2\,,\\ \varphi_6(x_1,x_2)&=x_1\oplus x_2\,,&
\varphi_7(x_1,x_2)&=x_1\lor x_2\,,&
\varphi_8(x_1,x_2)&=x_1\;\downarrow\;x_2\,,&\\
\varphi_9(x_1,x_2)&=x_1\sim x_2\,,& \varphi_{10}(x_1,x_2)&=\overline
x_2\,,& \varphi_{11}(x_1,x_2)&=x_1\leftarrow x_2\,,\\
\varphi_{12}(x_1,x_2)&=\overline x_1\,,&
\varphi_{13}(x_1,x_2)&=x_1\rightarrow x_2\,,&
\varphi_{14}(x_1,x_2)&=x_1\mid x_2\,,\\ \varphi_{15}(x_1,x_2)&=1\,.
  \end{align*}
\end{center}

%-------------------------------------------------------------------------------
\subsection{Разложение логических функций по переменным}

$$
\left.
\begin{aligned} &x^0=\overline x\,,\\ &x^1=x\,,\\
&\alpha\in{0,1}=B_2\,.
\end{aligned} \right\} \Rightarrow
\begin{aligned} x^\alpha&=1,& \text{если $x=\alpha$,}\\ x^\alpha&=0,&
\text{если $x\ne\alpha$.}
\end{aligned}
$$

\begin{theorem}[о разложении функций по переменным] Всякая логическая
функция $f(x_1,x_2,\ldots,x_n)$ может быть представлена в виде:

  \begin{equation}
    \label{logicfunc} \bigvee_{\left<
\alpha_1,\alpha_2,\ldots,\alpha_m \right> \subseteq
B_2^m}x_1^{\alpha_1},x_2^{\alpha_2},\ldots,x_m^{\alpha_m}
f(\alpha_1,\alpha_2,\ldots,\alpha_m,x_{m+1},\ldots,x_n)\,,
  \end{equation}
  
  где $m\leqslant n$, и дизъюнкция берется по всем $2^m$ наборам
$\left< \alpha_1,\alpha_2,\ldots,\alpha_m \right>$ значений переменных
$x_1,x_2,\ldots,x_n$.
\end{theorem}

\begin{ex} Разложим логическую функцию $f(x_1,x_2,x_3,x_4)$ по
переменным $x_1,x_2$, \linebreak[4] $m=2$\,:
  \begin{multline*} f(x_1,x_2,x_3,x_4) = \\ =
x_1^0\cdot{}x_2^0\cdot{}f(0,0,x_3,x_4) \lor
x_1^0\cdot{}x_2^1\cdot{}f(0,1,x_3,x_4) \lor
x_1^1\cdot{}x_2^0\cdot{}f(1,0,x_3,x_4) \lor
x_1^1\cdot{}x_2^1\cdot{}f(1,1,x_3,x_4) = \\ =
\overline{x_1}\cdot{}\overline{x_2}\cdot{}f(0,0,x_3,x_4) \lor
\overline{x_1}\cdot{}x_2\cdot{}f(0,1,x_3,x_4) \lor
x_1\cdot{}\overline{x_2}\cdot{}f(1,0,x_3,x_4) \lor
x_1\cdot{}x_2\cdot{}f(1,1,x_3,x_4)
  \end{multline*}
\end{ex}

\begin{proof} Подставим в обе части равенства~\eqref{logicfunc}
произвольный набор \linebreak $\left<
\sigma_1,\sigma_2,\ldots,\sigma_n
\right>$\,. Т.\,к. $x_i^{\alpha_i}=1$, то среди $2^m$ конъюнкций
$x_1^{\alpha_1}x_2^{\alpha_2}\ldots{}x_m^{\alpha_m}$ в единицу
обратится только одна, причём та, в которой
$x_1^{\alpha_1}=x_2^{\alpha_2}=\ldots=x_m^{\alpha_m}=1$\,, т.\,е. в
которой $x_1=\alpha_1$, $x_2=\alpha_2$, \ldots, $x_m=\alpha_m$\,. Это
означает, что:
  $$
  f(\sigma_1, \sigma_2, \ldots, \sigma_n) = \sigma_1^{\alpha_1}
\sigma_2^{\alpha_2} \ldots \sigma_m^{\alpha_m} f(\sigma_1, \sigma_2,
\ldots, \sigma_m, x_{m+1}, \ldots, x_n) \,.
  $$
\end{proof}

\begin{conseq}[СДНФ] Разложение логической функции по всем переменным
и формирование СДНФ:

  \begin{equation} f(x_1,x_2,\ldots,x_n)=
\bigvee_{f(\sigma_1,\sigma_2,\ldots,\sigma_n)=1}
x_1^{\sigma_1}x_2^{\sigma_2},\ldots,x_n^{\sigma_n}
  \end{equation}
\end{conseq}

\begin{ex} $f(x_1, x_2, x_3)$\,:\\ \\
  \begin{tabular}{c|c|c||c} $x_1$ & $x_2$ & $x_3$ & $f$\\ \hline
\hline 0 & 0 & 0 & 0\\ \hline 0 & 0 & 1 & 1\\ \hline 0 & 1 & 0 & 0\\
\hline 0 & 1 & 1 & 0\\ \hline 1 & 0 & 0 & 1\\ \hline 1 & 0 & 1 & 1\\
\hline 1 & 1 & 0 & 0\\ \hline 1 & 1 & 1 & 0\\
  \end{tabular} \qquad $ f(x_1, x_2, x_3) =
\overline{x_1}\,\overline{x_2}\,x_3 \lor
x_1\,\overline{x_2}\,\overline{x_3} \lor x_1\,\overline{x_2}\,x_3\,. $
\end{ex}

\begin{conseq}[СКНФ] Разложение логической функции по всем переменным
и формирование СКНФ:

  \begin{equation} f(x_1,x_2,\ldots,x_n)=
\bigwith_{f(\sigma_1,\sigma_2,\ldots,\sigma_n)=0}
x_1^{\sigma_1}x_2^{\sigma_2},\ldots,x_n^{\sigma_n}
  \end{equation}
\end{conseq}

%-------------------------------------------------------------------------------

\subsection{Булева алгебра. Свойства булевой алгебры}

\begin{defin}[булевой алгебры]
  $$ 
  A = (P_2;\lor,\with,\lnot)
  $$
  \begin{enumerate}
  \item Ассоциативность относительно:
    \begin{itemize}
    \item конъюнкции: $x_1 \with (x_2 \with x_3) = (x_1 \with x_2)
\with x_3$\,;
    \item дизъюнкции: $x_1 \lor (x_2 \lor x_3) = (x_1 \lor x_2) \lor
x_3$\,.
    \end{itemize}
  \item Коммутативность относительно:
    \begin{itemize}
    \item конъюнкции: $x_1 \with x_2 = x_2 \with x_1$\,;
    \item дизъюнкции: $x_1 \lor x_2 = x_2 \lor x_1$\,.
    \end{itemize}
  \item Дистрибутивность:
    \begin{itemize}
    \item конъюнкции относительно дизъюнкции: $x_1 \with (x_2 \lor
x_3) = x_1 \with x_2 \lor x_1 \with x_3$\,;
    \item дизъюнкции относительно конъюнкции: $x_1 \lor (x_2 \with
x_3) = (x_1 \lor x_2) \with (x_1 \lor x_3)$\,.
    \end{itemize}
  \item Идемпотентность:
    \begin{itemize}
    \item конъюнкции: $x \with x = x$\,;
    \item дизъюнкции: $x \lor x = x$\,.
    \end{itemize}
  \item $\lnot\lnot x = x$\,.
  \item Свойство констант:
    \begin{itemize}
    \item $x \with 1 = x$\,;
    \item $x \with 0 = 0$\,;
    \item $x \lor 1 = 1$\,;
    \item $x \lor 0 = x$\,;
    \item $\lnot 0 = 1$\,.
    \end{itemize}
  \item Правила де Моргана:
    \begin{itemize}
    \item $\overline{x_1 \with x_2} = \overline{x_1} \lor
\overline{x_2}$\,;
    \item $\overline{x_1 \lor x_2} = \overline{x_1} \with
\overline{x_2}$\,.
    \end{itemize}
  \item Закон противоречия: $x \with \overline{x} = 0$\,.
  \item Закон исключения 3-его: $x \lor \overline{x} = 1$\,.
  \end{enumerate}
\end{defin}

%-------------------------------------------------------------------------------
\subsection{Эквивалентные преобразования в булевой алгебре}
\begin{enumerate}
\item Свойство поглощения:
  \begin{itemize}
  \item $x \lor xy = x$\,;
  \item $x \with (x \lor y) = x$\,.
  \end{itemize}
\item Свойство склеивания и расщепления: $xy \lor x\overline{y} = x$\,.
\item Обобщённое склеивание: $xz \lor y\overline{z} \lor xy = xz \lor
y\overline{z}$.
\item $x \lor \overline{x}y = x \lor y$\,.
\item Обобщение: $x_1 \lor f(x_1,x_2,\ldots,x_n) = x_1 \lor
f(0,x_2,\ldots,x_n)$\,.
\end{enumerate}

%-------------------------------------------------------------------------------
\subsection{Интерпретация формул исчисления высказываний}

Как уже рассматривалось ранее, в исчислении высказываний каждой
формуле должно быть приписано истинностное значение или элемент
множества:
$$ \left\{ И,Л \right\} = \left\{ T,F \right\} = \left\{ 1,0 \right\}
\,.$$

Приписывая каждому атомарному символу из формулы $F$ истинностные
значения, мы будем получать и различные истинностные значения и самой
формулы $F$\,.

Таблицу, в которой указаны истинностные значения $F$ при всевозможных
истинностных значениях входящих атомов, назовём \emph{таблицей
истинности}.

\begin{ex} $F = (A \lor B) \supset C$
  \begin{center}
    \begin{tabular}{c|c|c||c|c|} $A$ & $B$ & $C$ & $A \lor B$ & $F$\\
\hline\hline 0 & 0 & 0 & 0 & 1\\ \hline 0 & 0 & 1 & 0 & 1\\ \hline 0 &
1 & 0 & 1 & 0\\ \hline 0 & 1 & 1 & 1 & 1\\ \hline 1 & 0 & 0 & 1 & 0\\
\hline 1 & 0 & 1 & 1 & 1\\ \hline 1 & 1 & 0 & 1 & 0\\ \hline 1 & 1 & 1
& 1 & 1\\ \hline
    \end{tabular}
  \end{center}
\end{ex}

\begin{defin}[интерпретации] Пусть $F$ "--- некоторая формула
исчисления высказываний, $A_1,A_2,\ldots,A_n$ "--- её атомарные
символы. \emph{Интерпретацией} $F$ является такое приписывание
истинностных значений атомам $A_1,A_2,\ldots,A_n$, при котором каждому
$A_i$ приписывается либо 1 либо 0.
\end{defin}

Говорят, что $F$ истинна при некоторой интерпретации тогда и только
тогда, когда $F$ получает значение 1 в этой интерпретации; в противном
случае $F$ ложна при этой интерпретации.

%-------------------------------------------------------------------------------
\subsection{Логические следствия в исчислении высказываний}

В математической логике следует различать понятия выводимости и
логического следования.

\begin{ex}[логическое следование] Событие $A$ "--- студент пришёл на
экзамен. $B$ "--- вытащил билет, ответ на который он знает. $C$ "---
экзамен сдан успешно.
  \begin{center} $A \with B \Rightarrow C$,\qquad $C$ "--- логическое
следование.
  \end{center}
\end{ex}

\begin{defin} Пусть даны формулы: $F_1,F_2,\ldots,F_n,G$. Говорят, что
$G$ "--- логическое следствие $F_1,F_2,\ldots,F_n$ или $G$ следует из
$F_1,F_2,\ldots,F_n$ тогда и только тогда, когда для всякой
интерпретации $I$, в которой $I\colon
F_1\with{}F_2\with\ldots\with{}F_n = 1$,\quad$G = 1$. В этом случае
говорят, что $F_1,F_2,\ldots,F_n$ "--- посылки, а $G$ "--- следствие, и
обозначают этот факт
  \begin{equation} F_1,F_2,\ldots,F_n \models G\,.
  \end{equation}
\end{defin}

\begin{theorem} Пусть даны $F_1,F_2,\ldots,F_n$ и $G$, тогда $G$ есть
логическое следствие формул $F_1,F_2,\ldots,F_n$ тогда и только тогда,
когда формула $(F_1 \with F_2 \with\ldots\with F_n) \supset G$
является общезначимой.
\end{theorem}

\begin{theorem} Пусть даны $F_1,F_2,\ldots,F_n$ и $G$. Формула $G$
есть логическое следствие формул $F_1,F_2,\ldots,F_n$ тогда и только
тогда, когда $F_1 \with F_2 \with\ldots F_n \with \overline{G}$
противоречива.
\end{theorem}

\begin{ex} $F_1\colon P \supset Q$,\quad $F_2\colon
\overline{Q}$,\quad $G\colon \overline{P}$,\quad где $P$ и $Q$ "---
атомы. Доказать, $F_1,F_2 \models G$.
  \begin{center}
    \begin{tabular}{c|c||c|c|c|c|c|c|} $P$ & $Q$ & $P \supset Q$ &
$\overline{Q}$ & $F_1 \with F_2$ & $\overline{P}$ & $F_1 \with F_2
\supset G$ & $F_1 \with F_2 \with \overline{G}$\\ \hline\hline 0 & 0 &
1 & 1 & 1 & 1 & 1 & 0\\ \hline 0 & 1 & 1 & 0 & 0 & 1 & 1 & 0\\ \hline
1 & 0 & 0 & 1 & 0 & 0 & 1 & 0\\ \hline 1 & 1 & 1 & 0 & 0 & 0 & 1 & 0\\
\hline
    \end{tabular}
  \end{center}
\end{ex}

%===============================================================================
\section{Исчисление предикатов или теории первого порядка}

\subsection{Определение исчисления предикатов}

\emph{Предикатом} $P(x_1,x_2,\ldots,x_n)$ называется функция $P$,
ставящая в соответствие $M^m\rightarrow B$ или $P\colon \underbrace{M
\times M \times\ldots\times M}_{\text{m раз}} \rightarrow B$, где
$M$ "--- произвольное множество , в том числе и бесконечное, называемое
предметной областью, $B = \{0,1\}$ "--- булево множество,
$x_1,x_2,\ldots,x_m$ "--- предметные переменные.

Другими словами, $m$-местный предикат, определенный на множестве
$M$ "--- это двухзначная функция от $m$ аргументов, принимающих
значение в произвольном множестве $M$.

Таким образом, видно, что предикат отличается от высказывания наличием
предметных переменных, которые могут принимать значения из некоторой
предметной области. Другим отличием предикатов является наличие
кванторов.

\begin{ex} $\exists x P(x)$,\quad $\forall x \overline{P}(x)$\,.
\end{ex}

\begin{defin} Переход от предиката P(x)
  
  $$
  P(x) \rightarrow
  \begin{cases} \exists x P(x)\\ \forall x P(x)
  \end{cases}
  $$
  называется операцией \emph{связывания переменной}, или операцией
\emph{квантификации}, а переменная $x$ "--- связанной. Если переменная
в предикате не связана, то она свободная.
\end{defin}

\begin{defin} Выражение или формула, на которые распространяется
действие квантора, называется \emph{областью действия квантора}.
\end{defin}

$$FT = \left<U,L,S,R\right>$$

\begin{description}
  \item[Алфавит] исчисления предикатов включает:
    \begin{itemize}
    \item $X = \{x_i\}$ "--- множество предметных переменных $x_i \in
M$.
    \item $A = \{a_j\}$ "--- множество предметных констант $a_j \in M$.
    \item $P = \{P_1^1,P_2^1,\ldots,P_k^l,\ldots\}$ "--- множество
предикатных букв, где $l$ "--- арность (местность) $k$-го предиката.
    \item $F = \{f_1^1,f_2^1,\ldots,f_k^l,\ldots\}$ "--- множество
функциональных букв, где $l$ "--- арность (местность) $k$-ой функции.
    \item знаки логических связок: $\lor$, $\with$, $\lnot$,
$\supset$, \ldots.
    \item кванторы: $\exists$, $\forall$, \ldots.
    \item служебные: .\quad,\quad(\quad)\quad \ldots.
    \end{itemize}
  \item[Язык] определяется в 2 этапа:
    \begin{enumerate}
    \item Термы (элементы $M$), $M \rightarrow B$:
      \begin{itemize}
      \item $x_i \in X$, $a_j \in A$;
      \item если $f^n \in F$ "--- функциональная буква,
$t_1,t_2,\ldots,t_n$ "--- термы, то термом также является
$f^n(t_1,t_2,\ldots,t_n)$.
      \end{itemize}
    \item Формулы:
      \begin{itemize}
      \item если $p^l \in P$, а $t_1,t_2,\ldots,t_l$ "--- термы, то
$P^l(t_1,t_2,\ldots,t_l)$ "--- формула;
      \item если $F_1$, $F_2$ "--- формулы, то формулами также являются
$F_1 \lor F_2$, $\overline{F_1}$, $\overline{F_2}$,~\ldots.
        \begin{rem} Все переменные, входящие в таким образом
построенные формулы, являются \emph{свободными}.
        \end{rem}
      \item если $F(x)$ "--- некоторая формула, в которой $x$ "---
свободная переменная, то $\exists x F(x)$ и $\forall x F(x)$ "--- также
формулы.
        \begin{rem} Формулы, построенные таким образом, являются
связанными.
        \end{rem}
      \end{itemize}
    \end{enumerate}
  \item[Аксиомы] исчисления предикатов делятся на 2 группы:
    \begin{enumerate}
    \item $S_I$ и $S_{II}$ "--- аксиомы исчисления высказываний;
    \item предикатные аксиомы:
      \begin{enumerate}
      \item $\forall x F(x) \supset F(y)$;
      \item $F(y) \supset \exists x F(x)$.
      \end{enumerate}
    \end{enumerate}
  \item[Правила вывода] исчисления предикатов содержат:
    \begin{enumerate}
    \item Modus Ponens;
    \item Правило обобщения (введения квантора $\forall$): $\frac{F
\supset G(x)}{F \supset \forall x G(x)}$
    \item Правило введения $\exists$: $\frac{G(x) \supset F}{\exists x
G(x) \supset F}$
    \end{enumerate}
\end{description}

%-------------------------------------------------------------------------------
\subsection{Интерпретация формул исчисления предикатов}

Интерпретация формулы F в исчислении предикатов определяется
сопоставлением:
\begin{enumerate}
\item Каждой предметной константе некоторого элемента из $M$.
\item Каждому $n$-местному функциональному символу отображения
$f\colon M^n \rightarrow M$.
\item Каждому $m$-местному предикатному символу $P\colon M^m
\rightarrow B$.
\end{enumerate}

\begin{ex} $F\colon \forall x P(x) = Л$.

  $M = \{1,2\}$.

  \begin{center}
    \begin{tabular}{c||c|c|} $M$ & 1 & 2\\ \hline\hline $P(x)$ & И(1)
& Л(0)\\
    \end{tabular}
  \end{center}

  Формула $F$ ложна в данной интерпретации.
\end{ex}

\begin{ex} $F\colon \forall x (P(x) \supset Q(f(x),a))$. Определить
интерпретацию и определить истинность в данной интерпретации.

  $M = \{1,2\}$,\qquad $a = 1$,\quad $a \in M$.

  \begin{center}
    \begin{tabular}{c||c|c|} $M$ & 1 & 2\\ \hline\hline $f(n)$ & 2 &
1\\
    \end{tabular} \qquad
    \begin{tabular}{c||c|c|} $M$ & 1 & 2\\ \hline\hline $P(x)$ & Л(0)
& И(1)\\
    \end{tabular}
  \end{center}
  
  \begin{center}
    \begin{tabular}{c||c|} $(x,y)$ & $Q(x,y)$\\ \hline\hline $(1,1)$ &
И(1)\\ \hline $(1,2)$ & И(1)\\ \hline $(2,1)$ & Л(0)\\ \hline $(2,2)$
& И(1)\\ \hline
    \end{tabular}
  \end{center}
\end{ex}

%-------------------------------------------------------------------------------

\subsection{Эквивалентные преобразования в исчислении предикатов}

\begin{defin} Формулы называются эквивалентными, если при любых
интерпретациях они принимают одно значение.
\end{defin}

\begin{enumerate}
\item $\overline{\exists x P(x)} = \forall x \overline{P(x)}$;
\item $\overline{\forall x P(x)} = \exists x \overline{P(x)}$;

\item[] Дистрибутивность операции $\forall$ относительно конъюнкции:
\item $\forall x (P_1(x) \with P_2(x)) = \forall x P_1(x) \with
\forall x P_2(x)$;

\item[] Дистрибутивность квантора существования относительно
дизъюнкции:
\item $\exists x (P_1(x) \lor P_2(x)) = \exists x P_1(x) \lor \exists
x P_2(x)$;

\item[] Односторонние:
\item $\exists x (P_1(x) \with P_2(x)) \Rightarrow \exists x P_1(x)
\with \exists x P_2(x)$;
\item $\forall x P_1(x) \lor \forall x P_2(x) \Rightarrow \forall x
(P_1(x) \lor P_2(x))$;

\item[] Коммутативность одинаковых кванторов:
\item $\forall x \forall y P(x,y) = \forall y \forall x P(x,y)$;
\item $\exists x \exists y P(x,y) = \exists y \exists x P(x,y)$;

\item[] Пусть $Y$ "--- некоторая формула, не содержащая переменную $x$:
\item $\forall x (P(x) \with Y) = \forall x P(x) \with Y$;
\item $\forall x (P(x) \lor Y) = \forall x P(x) \lor Y$;
\item $\exists x (P(x) \with Y) = \exists P(x) \with Y$;
\item $\exists x (P(x) \lor Y) = \exists x P(x) \lor Y$;
\end{enumerate}

\begin{rem} Поскольку исчисление предикатов есть расширение исчисления
высказываний, то для формул исчисления предикатов остаются верными и
эквивалентные предикаты, действующие для исчисления высказываний.
\end{rem}

%-------------------------------------------------------------------------------
\subsection{Нормальная форма в исчислении предикатов}

В исчислении предикатов так же, как и в исчислении высказываний,
имеются \emph{нормальные формы} (НФ), в частности, ПНФ "--- пренексная
нормальная форма:
\begin{equation} (Q_1x_1Q_2x_2 \ldots Q_nx_n)C\,,
\end{equation} где $Q_1,Q_2,\ldots,Q_n \in \{\exists,\forall\}$,
$(Q_1x_1,Q_2x_2,\ldots,Q_nx_n)$ "--- префикс, $C$ "--- матрица-формула в
КНФ (бескванторное логическое выражение в КНФ).

Любая формула исчисления предикатов может быть преобразована в ПНФ с
помощью некоторого алгоритма. Рассмотрим работу этого алгоритма на
примере:
\begin{ex} $F\colon \forall{x}(P_1(x)) \supset \lnot\forall{x}(P_2(y)
\lor \exists{y}P_3(x,y))$. Задача: построить ПНФ.
  \begin{description}
    \item[1 шаг:] поскольку КНФ предполагает использование логических
связок $\lor$, $\with$, $\lnot$, то нужно избавиться от $\supset$:\\
$A \supset B = \overline{A} \lor B$\,;\\
$\lnot\forall{x}\bigl(P_1(x)\bigr) \lor \lnot\forall{x}\bigl(P_2(y)
\lor \exists{y}P_3(x,y)\bigr)$\,.
    \item[2 шаг:] изменить (уменьшить) область действия $\lnot$:\\
$\exists{x}\bigl(\overline{P_1}(x)\bigr) \lor
\exists{x}\overline{\bigl(P_2(y)\lor\exists{y}P_3(x,y)\bigr)} =
\exists{x}\overline{P_1}(x) \lor
\exists{x}\bigl(\overline{P_2}(y)\with\lnot\exists{y}P_3(x,y)\bigr)
=$\\ $=\exists{x}\overline{P_1}(x) \lor \exists{x}\overline{P_2}(y)
\with \forall{y}\overline{P_3}(x,y)$\,.
    \item[3 шаг:] Переобозначим необходимые переменные для того, чтобы
не было коллизий (или не было одинаковых):\\
$\exists{u}\overline{P_1}(u) \lor \exists{x}\overline{P_2}(z) \with
\forall{y}\overline{P_3}(x,y)$\,.
    \item[4 шаг:] Формирование префикса для ПНФ:\\
$\exists{u}\overline{P_1}(u) \lor
\exists{x}\forall{y}\bigl(\overline{P_2}(z) \with
\overline{P_3}(x,y)\bigr) = \exists{u}\Bigl(\overline{P_1}(u) \lor
\exists{x}\forall{y}\bigl(\overline{P_2}(z) \with
\overline{P_3}(x,y)\bigr)\Bigr)=$\\ $=\exists{u}\exists{x}
\Bigl(\overline{P_1}(u) \lor \forall{y}\bigl(\overline{P_2}(z) \with
\overline{P_3}(x,y)\bigr)\Bigr) = \exists{u}\exists{x}\forall{y}
\bigl(\overline{P_1}(u) \lor \overline{P_2}(z) \with
\overline{P_3}(x,y)\bigr)$\,.
    \item[5 шаг:] Построение матрицы $C$ (КНФ):\\
$\exists{u}\exists{x}\forall{y}\Bigl(\bigl(\overline{P_1}(u) \lor
\overline{P_2}(z)\bigr) \with \bigl(\overline{P_1}(u) \lor
\overline{P_3}(x,y)\bigr)\Bigr)$\,.
  \end{description}
\end{ex}

%-------------------------------------------------------------------------------

\subsection{Доказательство теорем в исчислении предикатов}

Доказательство формул исчисления предикатов путём подстановки в них
констант называется \emph{методом интерпретации}. Этот метод позволяет
интерпретировать формулу как осмысленное утверждение, в связи с чем
такой метод называется \emph{семантическим}. Этот метод применим в
исчислении предикатов, вогда предметная область $M$ конечна. В таком
случае $M$ содержит конечное число элементов $M =
\{a_1,a_2,\ldots,a_n\}$.

В соответствии с семаническим подходом (методом), формула вида
$\forall{x}P(x)$ преобразуется в следующую ($x \in M$):
\begin{gather*} \forall{x}P(x) = P(a_1) \with P(a_2) \with \ldots
\with P(a_m)\\ \exists{x}P(x) = P(a_1) \lor P(a_2) \lor \ldots \lor
P(a_m)
\end{gather*}

Истинность таким образом преобразованных формул может быть проверена
путем конечного количества подстановок и вычислений. В случае же,
когда предметная область $M$ бесконечна ($|M| = \infty$),
семантический метод неприемлем. В этом случае используют процедуры
логического вывода с применением аксиом и правил вывода. Такой подход
называется \emph{формальным}.

%===============================================================================

\section{Метод резолюций и его использование в системах искусственного
интеллекта}

В этом разделе будут рассмотрены процедуры (алгоритмы) поиска
доказательств, которые могут быть использованы при проектировании
прикладных систем искусственного интеллекта.

%-------------------------------------------------------------------------------

\subsection{История поиска процедур доказательства теорем}

Проблема: разработка общего алгоритма для проверки общезначимости
формул исчисления высказываний и исчисления предикатов.

\begin{description}
  \item[1646--1716~гг.] Лейбниц, первые алгоритмы.
  \item[XVI век.] Пеан\'{о}.
  \item[1920-е гг.] Д.\,Гильберт.
  \item[1930-е гг.] А.\,Чёрч, А.\,Тьюринг независимо доказали, что не
существует общей решающей процедуры (алгоритма), проверяющей
общезначимость формул в исчислении предикатов. Тем не менее,
существуют алгоритмы поиска доказательств, которые могут подтвердить
общезначимость формулы, если она общезначима.
  \item[1930 г.] Эрбран Жак разработал алгоритм нахождения
интерпретации, которая опровергает заданную формулу, однако, если
формула общезначима, то такой интерпретации не существует, и алгоритм
заканчивает работу за конечное число ходов. Этот подход лёг в основу
современных методов.
  \item[1960 г.] Гилмор "--- попытка реализации метода Эрбрана на ЭВМ,
исходя из тезиса: <<формула общезначима тогда и только тогда, когда её
отрицание противоречиво>>. Программа Гилмора обнаруживала
противоречивость данной формулы, а значит, подтверждала её
общезначимость.
  \item[1960-е гг.] Если отрицание формулы противоречиво, то программа
обнаруживала этот факт, но была очень громоздкой. Дэвис и Патнем
доказали чрезмерную громоздкость и неприменимость в практике программы
Гилмора.
  \item[1965 г.] Робинсон Джон предложил наиболее эффективный алгоритм
поиска доказательства, который оказался самым эффективным. Сейчас
применяются много оптимизированных для разных областей вариаций этого
алгоритма:
    \begin{itemize}
    \item семантическая резолюция;
    \item лок-резолюция;
    \item линейная резолюция;
    \item алгоритм британского музея;
    \item и др.
    \end{itemize}
\end{description}

%-------------------------------------------------------------------------------

\subsection{Метод резолюций для исчисления высказываний}

\begin{theorem} Множество дизъюнктов $S=\{D_1,D_2,\dots,D_n\}$ "--- это
эквивалентная формулировка исходной теоремы:
  $$F = (F_1 \with F_2 \with \ldots \with F_n \supset \overline{G})\,,$$
  у которой $F_1,F_2,\ldots,F_n$ "--- посылки, $G$ "--- заключение, а
доказательством этой теоремы является опровержение (нахождение
опровержения) $F$.
\end{theorem}

\begin{proof} Доказательство теоремы сводится к доказательству
невыполнимости $S$. Если ввести в рассмотрение пустой дизъюнкт
$\square \equiv Л$, то сказанное выше интерпретируется следующим
образом: если среди элементов $S$ имеется хотя бы 1 пустой дизъюнкт,
то множество $S$ будет опровергаться, причём в любой интерпретации,
следовательно, $S$ невыполнимо.
\end{proof}

Если к невыполнимом множеству $S$ добавить некоторое подмножество $S'$
дизъюнктов таких, что $S'\colon S \supset S'$, то невыполнимость $S$ и
$S'$ будет сохранена.

Если среди логических следствий $S$ появится пустой дизъюнкт $S \lor
S'$, то это означает, что $S$ невыполнимо, а это в свою очередь
значит, что целью доказательства теоремы, сформулированной выше,
является получение среди логических следствий множества $S$ пустого
дизъюнкта.

\begin{defin}[резольвенты] Для любых 2-х дизъюнктов $C_1$ и $C_2$,
если существует литера $L_1$ в $C_1$, которая контрарна некоторой
литере $L_2$ в $C_2$, то, вычеркнув $L_1$ и $L_2$ из $C_1$ и $C_2$
соответственно, можно построить дизъюнкцию оставшихся
дизъюнктов. Построенный таким образом дизъюнкт называется
\emph{резольвентой} $C_1$ и $C_2$.
\end{defin}

\begin{ex} Рассмотрим $C_1=P\lor{}R$, $C_2=\overline{P}\lor{}Q$.
  $$ \left.
  \begin{aligned} C_1 &= P \lor R\\ C_2 &= \overline{P} \lor Q
  \end{aligned} \right\} \quad C = R \lor Q\text{ "--- резольвента.}
  $$
\end{ex}

Важным свойством резольвенты является то, что она является следствием
$C_1$ и $C_2$.

\begin{theorem} Пусть даны 2 дизъюнкта: $C_1$ и $C_2$. Тогда
резольвента $C$ дизъюнктов $C_1$ и $C_2$ есть логическое следование
исходных дизъюнктов, а именно:
  \begin{equation} (C_1 \with C_2) \supset C\,.
  \end{equation}
\end{theorem}

\begin{defin}[правило резолюций или правило вывода логических
следствий или правило резолютивного вывода] Пусть $S$ "--- некоторое
множество дизъюнктов. Резолютивный вывод $C$ из множества дизъюнктов
есть последовательность $C_1,C_2,\ldots,C_n$ дизъюнктов, каждый из
которых ($C_i$) либо принадлежит $S$, либо является резольвентой
дизъюнктов, предшествующих $C_i$.

  Вывод пустого дизъюнкта $\square$ из $S$ называется опровержением,
или доказательством невыполнимости $S$.
\end{defin}

\begin{ex} $S = \{\overline{P}\lor{}Q,\; \overline{Q},\; P\}$

  Резольвента $\overline{P}\lor{}Q$ и $\overline{Q}$ $\Rightarrow$
$\overline{P}$

  Резольвента $P$ и $\overline{P}$ $\Rightarrow$ $\square$

  Из $S \supset \square$ следует, что $S$ невыполнима.
\end{ex}

Таким образом, резольвента строится для 2-х дизъюнктов, содержащих
контрарную пару. Если для формулы исчисления высказываний такие пары
находятся (описываются) достаточно просто, то для формул исчисления
предикатов такая задача решается сложнее. Это связано с тем, что в
формулах исчисления предикатов необходимо ещё и совпадение литер
контрарной пары.

Т.\,к. задача резолютивного вывода "--- поиск опровержения исходной
формулы, то для этого можно использовать и константно-частные случаи
соответствующих дизъюнктов.

Поиск константно-частных случаев для дизъюнктов в формулах исчисления
предикатов, является предметом алгоритма унификации.

%%% Local Variables: 
%%% mode: latex
%%% TeX-master: "lections"
%%% End: 

\chapter{Экспертные системы}

\section{Введение в экспертные системы}

Цель исследований по экспертным системам (ЭС) состоит в разработке
программ, которые при решении задач, трудных для эксперта-человека,
получают результаты, не уступающие по качеству и эффективности
решениям, получаемым экспертом. Часто экспертные системы характеризуют
такую предметную область, для которой применим термин <<инженерия
знаний>>, под которым понимается привнесение принципов и
инструментария исследований из области искусственного интеллекта в
решение трудных прикладных проблем, требующих знаний эксперта.

Актуальность ЭС:
\begin{enumerate}
\item[1)] технология ЭС существенно расширяет круг практически значимых
  задач, решаемых на компьютерах, причём решение этих задач приносит
  значительный экономический эффект;
\item[2)] технология ЭС является важнейшим средством в решении глобальных
  проблем традиционного программирования, а именно: длительность и
  высокая стоимость разработки сложных приложений;
\item[3)] объединение технологий ЭС с технологией традиционного
  программирования добавляет новые качества к программам:
  \begin{itemize}
  \item обеспечение динамичной модификации приложений пользователем;
  \item большей <<прозрачности>> приложений;
  \item лучшей графики, интерфейса и пр.
  \end{itemize}
\end{enumerate}

По мнению ведущих специалистов в будущем ЭС найдут следующее
применение:
\begin{enumerate}
\item[1)] ЭС будут играть ведущую роль на всех фазах проектирования,
  разработки, производства, сопровождения, поддержки, распределения и
  оказания различных услуг.
\item[2)] технология ЭС, получившая коммерческое распространение,
  обеспечит революционный прорыв в интеграции приложений из готовых
  интеллектуальных и информационно взаимодействующих модулей.
\end{enumerate}

ЭС предназначены для решения так называемых <<слабоформализуемых>> или
<<неформализованных>> задач, т.\,е. ЭС не отвергают и не заменяют
традиционные технологии программирования.

Слабоформализуемые задачи обладают следующими особенностями:
\begin{enumerate}
\item[1)] ошибочностью, неоднозначностью, неполнотой и
  противоречивостью исходных данных и знаний;
\item[2)] ошибочностью, неоднозначностью, неполнотой и
  противоречивостью знаний о предметной области и методах решения
  задач;
\item[3)] большой размерностью результата вычислений, т.\,е. прямой
  перебор результатов возможных решений очень затруднён;
\item[4)] динамичностью изменяющихся данных и знаний.
\end{enumerate}

\begin{rem}
  Следует подчеркнуть, что слабоформализуемые задачи представляют
  собой очень большой и важный класс задач. Более того, многие
  считают, что эти задачи являются наиболее массовым классом задач,
  требующих решения на ЭВМ.
\end{rem}

Решения ЭС обладают свойством прозрачности, т.\,е. могут быть
объяснены пользователю на качественном уровне (объяснительная
возможность ЭС). Это качество ЭС обеспечивается их способностью
рассуждать о своих знаниях и умозаключениях.

Необходимо отметить, что в настоящее время технология ЭС используется
для решения задач следующих типов:
\begin{itemize}
\item интерпретация знаний;
\item предсказание событий;
\item диагностирование;
\item планирование;
\item конструирование;
\item контроль функционирования;
\item управление в различных областях:
  \begin{itemize}
  \item финансы;
  \item нефтяная и газовая промышленность;
  \item транспорт;
  \item авиация и космонавтика;
  \item металлургия;
  \item и прочих.
  \end{itemize}
\end{itemize}

Причины, приведшие СИИ (ЭС) к коммерческому успеху:
\begin{enumerate}
\item[1)] интегрированность: разрабатываемые средства легко
  интегрируются с другими технологиями и средствами СУБД, CASE и др.;
\item[2)] открытость и переносимость;
\item[3)] использование традиционных языков программирования;
\item[4)] использование архитектуры <<клиент-сервер>>, технологии
  промежуточного программного обеспечения.
\end{enumerate}

%===============================================================================

\section{Структура экспертных систем}

\subsection{ЭС статического типа}

\begin{description}
\item[БД (рабочая память)] предназначена для хранения исходных и
  промежуточных данных решаемой в текущий момент задачи. Данный термин
  совпадает с термином, используемым в информационно-поисковых
  системах и СУБД для обозначения всех данных (в первую очередь
  краткосрочных), хранимых в системе.
\item[База знаний (БЗ)] предназначена для хранения долгосрочных данных
  и знаний, описывающих рассматриваемую ПрО, а не текущих данных (БД),
  а также правил, описывающих целесообразные преобразования хранимых
  здесь данных и знаний.
\item[Решатель,] используя исходные данные из рабочей памяти и знания
  из базы знаний, формирует такую последовательность правил, которые,
  будучи применёнными к исходным данным, приводят к решению задачи.
\item[Система приобретения знаний] решает задачи по
  автоматизированному наполнению ЭС знаниями и предназначена для
  применения конечным пользователем (не программистом).
\item[Объяснительная система] объясняет, каким образом ЭС получила то
  или иное решение или не получила решения и какие знания система при
  этом использовала. Применение этой системы облегчает конечному
  пользователю тестирование системы и повышает доверие к полученному
  результату.
\item[Диалоговая система]  ориентирована на организацию дружественного
  общения с конечным пользователем как на этапе приобретения знаний,
  так и на этапе получения и объяснения результатов (решений).
\end{description}

Описанная система является статической, поскольку используется в тех
приложениях, где можно не учитывать изменения окружающего мира в
процессе получения решения.

\subsection{ЭС динамического типа}

В архитектуру динамической ЭС по сравнению со статической вводятся
дополнительно 2 компонента:
\begin{enumerate}
\item[1)] подсистема моделирования внешнего мира;
\item[2)] подсистема сопряжения с внешней средой.
\end{enumerate}

Подсистема сопряжения с внешним миром осуществляет связь в внешним
миром через набор датчиков и контроллеров.

Традиционные компоненты статических ЭС в динамических ЭС, прежде
всего БЗ и решатель, претерпевают значительные изменения, чтобы
отразить временную логику происходящих в реальном мире событий.

%===============================================================================

\section{Интеллектуальные информационные технологии и развитие
  аппарата знаний}

До последнего времени состояние исследований в развитии аппарата
знаний соответствовало следующей схеме:

Основные направления развития аппарата знаний:
\begin{enumerate}
\item Извлечение знаний из различных источников (формализация и
  интерпретация знаний.
\item Приобретение знаний от профессионалов.
\item Представление знаний:
  \begin{enumerate}
  \item Модели знаний:
    \begin{enumerate}
    \item Семантические сети.
    \item Фреймы (сети фреймов).
    \item Логические системы.
    \item Продукции.
    \end{enumerate}
  \item Системы представления знаний.
  \item Базы знаний.
  \end{enumerate}
\item Манипулирование знаниями.
\item Объяснение знаний.
\end{enumerate}

%===============================================================================

\section{Модели представления знаний (МПЗ)}

Типы МПЗ:
\begin{itemize}
\item Декларативные:
  \begin{itemize}
  \item продукционные;
  \item редукционные;
  \item предикатные.
  \end{itemize}
\item Процедурные:
  \begin{itemize}
  \item PLANNER;
  \item CONNIVER;
  \item ПРИЗ.
  \end{itemize}
\item Специальные (комбинированные):
  \begin{itemize}
  \item семантические сети;
  \item сети фреймов;
  \item нечёткие МПЗ;
  \item реляционные МПЗ.
  \end{itemize}
\end{itemize}

%===============================================================================

\section{Аппарат знаний и его влияние на интеллектуализацию
  информационной технологии}

В настоящее время наблюдаются следующие основные тенденции названного
влияния:
\begin{enumerate}
\item \emph{Переход от классических вычислений (архитектура фон Неймана) к
  альтернативным способам организации вычислительного процесса.}

  В течение нескольких последних десятков лет постоянно велись
  исследования по переходу от архитектуры фон Неймана к иным способам,
  связанным, в основном, с исследованиями в области искусственного
  интеллекта и параллельного программирования для многопроцессорных
  систем. Эти исследования в последнее время реализуются на моделях,
  строящихся децентрализованными, асинхронными, максимально
  параллельными, управляемыми по данным в процессе вычисления.
\item \emph{Технология активных объектов.}

  Ключевой в перестройке информационных технологий последних лет
  явилась реализация подхода на основе объектно-ориентированного
  программирования (ООП), однако этот подход определил пока лишь
  фундамент будущей технологии, оставляя прежним алгоритмический
  характер управления процессом вычислений.

  Тем временем, направление развития по данным и дальнейшее развитие
  направлений на основе событий формируют следующие направления
  интеллектуализации информационных технологий:
  \begin{itemize}
  \item на основе автономных активных объектов, интегрирующих
    мультиагентную архитектуру;
  \item методы программирования в ограничениях;
  \item аппарат недоопределённых моделей.
  \end{itemize}
\item \emph{Приоритет модели, а не алгоритма
    (см. разд.~\ref{sec:models_and_algorithms})}.

  Известен прогноз, который предполагает, что через 10--15 лет
  алгоритм ожидает судьба ассемблера и машинных кодов и потеря
  сегодняшних ключевых позиций, а также места в сравнительно тонком
  базовом уровне информационных технологий будущего
\item \emph{Параллелизм.}

  В новых технологиях параллельность перестаёт быть проблемой, а
  становиться естественным свойством любой программной системы.
\end{enumerate}



%===============================================================================

\section{Смена парадигмы разработки интеллектуальных информационных систем}

В теории программирования существуют 2 противоположных и
взаимодополняющих друг друга понятия:
\begin{enumerate}
\item Императивное (алгоритмическое, командное) \textrightarrow{} понятие
  алгоритма;
\item Декларативное (непроцедурное, основанное на моделях)
  \textrightarrow{} данные.
\end{enumerate}

\subsection{История развития информационных технологий}

Под \emph{моделью} можно понимать определенное множество абстрактных объектов
(несколько множеств абстрактных объектов), различающихся условно
приписываемым им именам в совокупности с заданной системой отношений
между элементами этих множеств.

Под \emph{алгоритмом} можно понимать точное предписание
последовательности действий, необходимых для получения искомых результатов.

\begin{table}[ht]
  \centering

  \topcaption{Этапы развития информационных технологий}
  \label{tab:stages}

  \begin{tabular}{|*{6}{m{.14\linewidth}|}}
    % \begin{tabular}{|p{2.2cm}|p{2.2cm}|p{2.2cm}|p{2.2cm}|p{2.2cm}|p{2.2cm}|}
    \multicolumn{3}{c|}{Модели} &
    \multicolumn{3}{c}{Алгоритмы}\\
    \hline
    \multicolumn{6}{c}{Этап 1:}\\ \cline{4-6}
    \multicolumn{3}{c|}{} & Раздел данных & Операторы обработки & Операторы управления\\ \cline{4-6}
    \multicolumn{6}{c}{Этап 2:}\\ 
    \cline{2-5}
    \multicolumn{1}{c|}{}&Данные & Модели данных & Операторы обработки & Операторы
    управления\\ \cline{2-5}
    \multicolumn{6}{c}{Этап 3:}\\ 
    \cline{1-4} Данные & Модели данных & Модели знаний & Опе\-ра\-ци\-он\-ная среда
    (решатель)\\ \cline{1-4}
  \end{tabular}
\end{table}


Таким образом, если на 1-ом этапе развития информационных технологий
раздел данных можно рассматривать как неразвитый элемент в большом
теле алгоритма, то интеллектуальный решатель на 3-ем этапе
представляет собой небольшой автономный элемент алгоритма в теле моделей.

%-------------------------------------------------------------------------------

\subsection{\label{sec:models_and_algorithms}Смена влияния моделей и алгоритмов на развитие
  информационных технологий}

В подтверждение сказанному выше, можно отметить следующее: на
начальном этапе развития вычислительной техники программирование в
кодах считалось единственно возможным способом общения с ЭВМ, однако
уже к настоящему времени этот тип программирования сохранил своё место
только в тонком, ближайшем к аппаратной части слое операционных
систем. По аналогии и модель завоёвывает своё место в практике
программирования. Именно модель представляет собой объект исследования
и определяет характер формального аппарата, используемого для описания
задач.

С моделью работает конечный пользователь --- специалист конкретной
предметной области. С алгоритмом же работает профессиональный
программист и разработчик вычислительных систем. Всё это определило
парадоксальность современного состояния прикладных информационных
технологий, когда модель можно встретить в основном только в
теоретических работах как иллюстрацию к объекту исследования.

%  \begin{tabular}{m{.45\linewidth}|m{.45\linewidth}}

\begin{table}[ht]
  \centering

  \topcaption{Качественное сопоставление модели и алгоритма}
  \label{tab:model_and_algorithm}

  \begin{tabular}{m{.45\linewidth}|m{.45\linewidth}} \hline
    \multicolumn{1}{c|}{\large\textbf{Модель}} &
    \multicolumn{1}{c}{\large\textbf{Алгоритм} \rule{0pt}{16pt}} \\[5pt]
    \hline \hline
    
    Принципиально декларативна & Антидекларативна\\ \hline
    
    Симметрична по отношению к своим параметрам, поскольку все они
    определяются друг через друга & Разделяет параметры на входные и
    выходные явным образом, определяя вторые через первые\\ \hline
    
    В неявной форме определяет решение всех задач, связанных с объектом
    исследования (программирования) & Определяет в явной форме и задаёт
    решение только одной задачи, отношение которой к реальному объекту не
    всегда очевидно\\ \hline
    
    Может быть недоопределённой & Алгоритм и недоопределённость "---
    несовместимые понятия\\ \hline
    
    В общем случае определяет всё пространство решений & Позволяет
    получать только отдельные точечные решения (в общем случае).\\ \hline
  \end{tabular}
\end{table}



Можно представить себе такую идеальную ситуацию, когда компьютер
способен взаимодействовать с конечным пользователем непосредственно
(напрямую). Получая на вход формальную модель или её описание,
компьютер автоматически <<сжимает>> её до минимального $k$-мерного
параллелепипеда в соответствующем гиперпространстве параметров
модели. При введении дополнительных ограничений или изменении
параметров модели этот параллелепипед в общем случае <<сжимается>> или
изменяет свои размеры (уточняется, дополняется), возможно исчезая
совсем, в том случае, если модель или введённые ограничения
несовместимы.

Такой идеальный способ организации вычислений на основе моделей в
форме сжатия пространства моделей был бы внутренне параллельным, а
также недетерминированным и асинхронным, и, следовательно,
естественным образом переносимым на параллельные ЭВМ.

\begin{rem}
  Очевидно, что для человека, сформировавшегося как программист или
  как конечный пользователь на основе императивной технологии,
  описанный подход представляется совершенно фантастическим и
  невозможным в принципе. Однако, такие технологии уже существуют, в
  частности, такие технологии обеспечивают пользователя почти всем
  спектром необходимых возможностей и позволяют работать с такими
  моделями без посредников, допуская в рамках одной модели сочетания
  различных формальных аппаратов (алгебры логики, функционального
  анализа, теории множеств) "--- принцип полимодельности.
\end{rem}

Подытожив сказанное, можно отметить, что на самом деле модель и
возможность прямого взаимодействия с нею являлись с самого начала
развития искусственного интеллекта ключевым ориентиром и естественным
следствием этих исследований была принципиальная потребность выхода за
пределы парадигмы алгоритма в самых различных направлениях:
\begin{itemize}
\item lisp;
\item prolog;
\item фреймы;
\item продукционные системы;
\item мультиагентные системы;
\item методы удовлетворения ограничений (недоопределённость
  вычислений).
\end{itemize}

Объективный прогноз развития информационных технологий говорит, что в
ближайшие 10--20лет алгоритм ожидает судьба ассемблера и
программирования в годах, потеря сегодняшних ключевых позиций и места
в сравнительно тонком базовом уровне будущей информационной технологии.

%%% Local Variables: 
%%% mode: latex
%%% TeX-master: "lections"
%%% End: 
\appendix
\chapter{Программный комплекс мониторинга состояний и управления
  сложными техническими объектами в реальном масштабе времени} 
\section{Назначение, характеристики и структура информационных систем
  мониторинга состояния сложных технических объектов}

В настоящее время состояние дел в области проектирования и
эксплуатации программных комплексов мониторинга состояния сложных
технических объектов может быть охарактеризовано рядом положений:
\begin{itemize}
\item большие (сверхбольшие) потоки обрабатываемой информации;
\item возрастание сложности объектов управления;
\item увеличение количества объектов управления;
\item большое разнообразие типов измерительной информации,
  используемой для принятия решений;
\item неопределённость и нечётность задач при проведении мониторинга
  задач сложных технических  объектов.
\end{itemize}

%===============================================================================

\section{Характеристика основных элементов информационной технологии,
  которую реализует программный комплекс}

\subsection{Назначение информационной технологии, реализуемой
  программным комплексом}

Мониторинг состояний (МС) предполагает получение в явном виде
обобщённых оценок выполнения программы функционирования
рассматриваемого объекта управления, либо степень его
работоспособности, места и вида возникшей неисправности, оценок
прогнозируемых процессов с заданной точностью, с учётом конкретных
целей и условий эксплуатации на различных этапах его функционирования.

Мониторинг может проводиться при интеграции всех имеющихся видов
измерительной информации и решать следующий перечень задач:
\begin{itemize}
\item \emph{контроль функционирования объекта управления},
  выполняющийся при его нахождении как штатных, так и внештатных
  ситуациях; при реализации технологии автоматизированного управления
  при мониторинге может быть реализован режим выдачи команд оператора
  с АРМ;
\item \emph{контроль работоспособности объекта управления} и, при
  возникновении неисправности, её диагностирование с указанием места и
  вида возникшей ситуации;
\item \emph{прогнозирование поведения объекта управления} и
  предсказание развития как штатных, так и нештатных (аварийных)
  ситуаций с целью их предупреждения и недопущения.
\end{itemize}

Основным элементом автоматизации процесса МС является функциональный
элемент этой системы, решающий задачу по сбору, обработке и анализу
всех видов измерительной информации для произвольного типа и
количества пользователей.

Таким функциональным элементом является унифицированный типовой модуль
автоматизации, объединяющий в себе систему поддержки интерфейса
че\-ло\-век-ма\-ши\-на и программное обеспечение операторских станций.

%===============================================================================
\subsection{Характеристика унифицированного типового модуля
  автоматизации (УТМА)}

При создании автоматизированных систем управления технологическими
процессами любой сложности всегда существовала тяжело решаемая
проблема: каким образом реализовать технологию формализации знаний о
функционировании объектов управления (ОУ). Неоднократно
предпринимались попытки реализовать такие технологии путём
использования специализированных языков (блок-схемы, циклограммы
функционирования и т.д.), однако все эти попытки были обречены на
провал. Причиной послужило то, что очень трудно (почти невозможно)
добиться от конструктора ОУ хотя бы словесного описания алгоритма
манипулирования данными при МС, и дажи при получении такого описания,
переложить его на язык программирования. Основная причина такого
положения --- принципиально разные языки, на которых общаются
программисты и технологи: программисты используют языки
программирования, технологи --- технологические языки описания
соответствующей предметной области.

Единственный выход из этой ситуации --- предоставить конечному
пользователю понятное ему средство для описания соответствующей
предметной области, объяснив при этом возможность самостоятельно (без
участия программиста) описывать технологические процессы
функционирования ОУ. Основное достоинство рассматриваемого УТМА
состоит в том, что с его помощью создаются (<<программируются>>)
уникальные модули автоматизации, причём самими технологами, почти без
участия программистов. Вся работа технолога состоит в том, что он
отлаживает непосредственно сам процесс технологического управления,
записанного на понятном пользователю языке и состоящем из примитивов,
которыми этот конечный пользователь оперирует. При этом технологу
необходимы такие основные примитивы для описания своей предметной
области, которые включают типовой набор функций, присущий всем
процессам автоматизации для мониторинга состояния соответствующих ОУ:
\begin{itemize}
\item экранные формы отображения значений, измеряемых параметров типа
  стрелочных, полосковых или цифровых индикаторов, а также
  сигнализирующие панели различной формы и содержания, например,
  индикатор температуры;
\item возможность создания архивов штатных и нештатных ситуаций,
  событий и поведения динамических процессов во времени (так
  называемые <<тренды>>), а также полное и выборочное сохранение
  измеряемых параметров по времени и по условию;
\item упрощенный, адаптированный к предметной области язык для
  реализации алгоритмов обработки информации, математических и
  логических выражений (например, язык схем анализа, используемый на
  лабораторных работах;
\item ядро (монитор реального времени), которое обеспечивает
  детерминизм поведения информационной системы или, иными словами,
  предскозуемое время отклика на внешнее событие. Существует понятие
  системы рельного времени. В такой системе жёстко задано максимальное
  время отклика, равное $\Delta t$;
\item драйверы управления различного рода внешним оборудованием;
\item средства, реализующие сетевые функции;
\item средства защиты от несанкционированного доступа;
\item многооконный графический интерфейс и другие очевидные функции,
  такие как импорт изображений, создание собственной библиотеки
  алгоритмов, динамических объектов и др.;
\item средства ведения протокола работы пользователя при проведении
  сеансов работы с данными;
\item средства обеспечения работы исполнительной системы на
  разнородных про\-грамм\-но-ап\-па\-рат\-ных платформах;
\item и др.
\end{itemize}

При обеспечении функционирования информационной системы этапность
работы конечного пользователя определяется следующей схемой:
\begin{enumerate}
\item Формирование статического изображения рабочего стола (окна):
  \begin{itemize}
  \item мнемосхема контролируемого объекта;
  \item информационные и управленческие связи на мнемосхеме;
  \item органы управления различных типов (кнопки, выключатели и др.);
  \item и др.
  \end{itemize}
\item Формирование динамических объектов рабочего стола:
  \begin{itemize}
  \item стрелочные индикаторы;
  \item полосковые индикаторы;
  \item цифровые индикаторы;
  \item тренды контролируемых процессов;
  \item произвольные сигнализирующие табло;
  \item и др.
  \end{itemize}
\item Описание взаимодействия между имеющимися этапами на мнемосхемах:
  \begin{itemize}
  \item динамические объекты взаимодействия;
  \item входные/выходные каналы измерительной и обработанной
    информации;
  \item и др.
  \end{itemize}
\end{enumerate}

%===============================================================================
\section{Режимы распределенной иерархической обработки данных}

В распределенных иерархических системах сбора и обработки данных
выделяются несколько уровней:
\begin{itemize}
\item уровень непосредственного сбора данных. Основан на использовании
  датчиков, регуляторов и исполнительных механизмов, на которых
  программное обеспечение загружается с ПЗУ в ОЗУ, флеш-память и др.
\item основной уровень. На нём собирается вся необходимая информация
  от источников низшего уровня и включает в свой состав не только
  вычислительные средства, но и человека-оператора.
\item уровень оценки состояния ОУ. На этом уровне выполняются операции
  по модернизированию, прогнозированию, оптимизации процессов, и на
  него поступает информация, собранная на основном уровне. При этом
  предполагается использование мощных вычислительных ресурсов. Модуль
  автоматизации на этом уровне строится на базе экспертных,
  расчётно-логических или моделирующих систем обработки данных
  рельного времени.
\end{itemize}

Типовой модуль автоматизации на любом из отмеченных уровней состоит
из:
\begin{enumerate}
\item Базы данных измерений (БДИ) и диалогового редактора баз данных.
\item Графического редактора статических и динамических изображений на
  мнемосхемах.
\item Графического редактора символов, который позволяет создавать
  библиотеки типовых пиктограмм, используемых графическими редакторами
  статических и динамических изображений.
\item Средств сбора и отображения данных в предыстории по любому
  параметру из базы данных.
\item Генератора отчётов, который позволяет формировать отчеты по
  данным рельного времени и по предыстории.
\item Средств отображения событий реального времени (циклограммы,
  мнемосхемы, тренды, свотки событий и тревог, звуковая и речевая
  сигнализация и др.).
\item Средств обработки значений параметров и вычислений, задаваемых
  пользователем по алгоритму.
\end{enumerate}

Распределённые системы информационного обеспечения включают следующие
компоненты:
\begin{itemize}
\item средства поддержки сетевой работы в рамках распределенной
  системы, как в составе ЛВС различных технологий, так и в составе
  ГВС;
\item средство обмена данными и сообщениями между операторами и
  рабочими станциями;
\item средства пароля и защиты, разграничения доступа по уровню прав
  оператора;
\item система <<горячего>> резервирования и автоматического
  восстановления для обеспечения надежности, устойчивости и
  непрерывности вычислительного процесса.
\end{itemize}

%===============================================================================
\section{Основные технологических элементы информационной технологии
  обработки данных}

Основным технологическим элементом рассматриваемой технологии является
программный комплекс, реализующий всю её функциональность. Структура
программного комплекса:
\begin{itemize}
\item операционная среда (система автоматизированной подготовки
  исходных данных и знаний);
\item измерительная среда (система автоматизированной обработки
  (анализа) измерительной информации);
\item система генерации и обслуживания программного комплекса.
\end{itemize}

Структура операционной среды включает в себя следующие базовые
элементы:
\begin{itemize}
\item монитор операционной среды;
\item операционную систему;
\item сетевые средства;
\item средства компонентных вычислений (для платформы Windows ---
  DM-системы, для UNIX --- CORBA).
\end{itemize}

Структура функциональных подсистем:
\begin{enumerate}
\item Система баз данных.
\item Специализированные интерактивные подсистемы:
  \begin{itemize}
  \item концептуальное моделирование;
  \item поведенческое моделирование;
  \item GUI;
  \item автоматический синтез корректной метапрограммы.
  \end{itemize}
\item Языковые средства операционной среды:
  \begin{itemize}
  \item подсистема описания измеряемых и вычисляемых параметров;
  \item подсистема описания групп параметров для их совместной
    обработки;
  \item подсистема описания моделей сегментации значений параметров;
  \item подсистема описания динамических моделей изменения значений
    параметров;
  \item подсистема описания диалоговых панелей отображения;
  \item подсистема запросов к базе данных;
  \item подсистема контроля функционирования объекта управления;
  \item подсистема проверки условий и изменения ходя вычислений;
  \item подсистема вызова автоматизированных операций над данными;
  \item подсистема графического (мультимедийного) отображения;
  \item подсистема организации диалога с конечным пользователем.
  \end{itemize}
\end{enumerate}

\begin{thebibliography}{9}
\bibitem{Efimov}
Ефимов Е.И. \textsl{Решатели интеллектуальных задач},
М., Наука, 1992.
\bibitem{Kuzin}
Кузин Л.Т. \textsl{Основы кибернетики},
1979.
\bibitem{Nilson}
Нильсон Н. \textsl{Принципы искусственного интеллекта},
М., Радиосвязь, 1985.
\bibitem{Simons}
Симонс Дж. \textsl{ЭВМ пятого поколения: компьютеры 90-х годов},
М., Статистика, 1985.
\bibitem{Endryu}
Эндрю А. \textsl{Искусственный интеллект},
М., Мир, 1985.
\bibitem{}
А. Тей, П. Грибомон, Ж. Луи, Д. Снийерс. \textsl{Логический подход к искусственному интеллекту: от классической логики к логическому программированию},
М., Мир, 1990.
\end{thebibliography}


\end{document} 
