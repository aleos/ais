\chapter{Экспертные системы}

\section{Введение в экспертные системы}

Цель исследований по экспертным системам (ЭС) состоит в разработке
программ, которые при решении задач, трудных для эксперта-человека,
получают результаты, не уступающие по качеству и эффективности
решениям, получаемым экспертом. Часто экспертные системы характеризуют
такую предметную область, для которой применим термин <<инженерия
знаний>>, под которым понимается привнесение принципов и
инструментария исследований из области искусственного интеллекта в
решение трудных прикладных проблем, требующих знаний эксперта.

Актуальность ЭС:
\begin{enumerate}
\item[1)] технология ЭС существенно расширяет круг практически значимых
  задач, решаемых на компьютерах, причём решение этих задач приносит
  значительный экономический эффект;
\item[2)] технология ЭС является важнейшим средством в решении глобальных
  проблем традиционного программирования, а именно: длительность и
  высокая стоимость разработки сложных приложений;
\item[3)] объединение технологий ЭС с технологией традиционного
  программирования добавляет новые качества к программам:
  \begin{itemize}
  \item обеспечение динамичной модификации приложений пользователем;
  \item большей <<прозрачности>> приложений;
  \item лучшей графики, интерфейса и пр.
  \end{itemize}
\end{enumerate}

По мнению ведущих специалистов в будущем ЭС найдут следующее
применение:
\begin{enumerate}
\item[1)] ЭС будут играть ведущую роль на всех фазах проектирования,
  разработки, производства, сопровождения, поддержки, распределения и
  оказания различных услуг.
\item[2)] технология ЭС, получившая коммерческое распространение,
  обеспечит революционный прорыв в интеграции приложений из готовых
  интеллектуальных и информационно взаимодействующих модулей.
\end{enumerate}

ЭС предназначены для решения так называемых <<слабоформализуемых>> или
<<неформализованных>> задач, т.\,е. ЭС не отвергают и не заменяют
традиционные технологии программирования.

Слабоформализуемые задачи обладают следующими особенностями:
\begin{enumerate}
\item[1)] ошибочностью, неоднозначностью, неполнотой и
  противоречивостью исходных данных и знаний;
\item[2)] ошибочностью, неоднозначностью, неполнотой и
  противоречивостью знаний о предметной области и методах решения
  задач;
\item[3)] большой размерностью результата вычислений, т.\,е. прямой
  перебор результатов возможных решений очень затруднён;
\item[4)] динамичностью изменяющихся данных и знаний.
\end{enumerate}

\begin{rem}
  Следует подчеркнуть, что слабоформализуемые задачи представляют
  собой очень большой и важный класс задач. Более того, многие
  считают, что эти задачи являются наиболее массовым классом задач,
  требующих решения на ЭВМ.
\end{rem}

Решения ЭС обладают свойством прозрачности, т.\,е. могут быть
объяснены пользователю на качественном уровне (объяснительная
возможность ЭС). Это качество ЭС обеспечивается их способностью
рассуждать о своих знаниях и умозаключениях.

Необходимо отметить, что в настоящее время технология ЭС используется
для решения задач следующих типов:
\begin{itemize}
\item интерпретация знаний;
\item предсказание событий;
\item диагностирование;
\item планирование;
\item конструирование;
\item контроль функционирования;
\item управление в различных областях:
  \begin{itemize}
  \item финансы;
  \item нефтяная и газовая промышленность;
  \item транспорт;
  \item авиация и космонавтика;
  \item металлургия;
  \item и прочих.
  \end{itemize}
\end{itemize}

Причины, приведшие СИИ (ЭС) к коммерческому успеху:
\begin{enumerate}
\item[1)] интегрированность: разрабатываемые средства легко
  интегрируются с другими технологиями и средствами СУБД, CASE и др.;
\item[2)] открытость и переносимость;
\item[3)] использование традиционных языков программирования;
\item[4)] использование архитектуры <<клиент-сервер>>, технологии
  промежуточного программного обеспечения.
\end{enumerate}

%===============================================================================

\section{Структура экспертных систем}

\subsection{ЭС статического типа}

\begin{description}
\item[БД (рабочая память)] предназначена для хранения исходных и
  промежуточных данных решаемой в текущий момент задачи. Данный термин
  совпадает с термином, используемым в информационно-поисковых
  системах и СУБД для обозначения всех данных (в первую очередь
  краткосрочных), хранимых в системе.
\item[База знаний (БЗ)] предназначена для хранения долгосрочных данных
  и знаний, описывающих рассматриваемую ПрО, а не текущих данных (БД),
  а также правил, описывающих целесообразные преобразования хранимых
  здесь данных и знаний.
\item[Решатель,] используя исходные данные из рабочей памяти и знания
  из базы знаний, формирует такую последовательность правил, которые,
  будучи применёнными к исходным данным, приводят к решению задачи.
\item[Система приобретения знаний] решает задачи по
  автоматизированному наполнению ЭС знаниями и предназначена для
  применения конечным пользователем (не программистом).
\item[Объяснительная система] объясняет, каким образом ЭС получила то
  или иное решение или не получила решения и какие знания система при
  этом использовала. Применение этой системы облегчает конечному
  пользователю тестирование системы и повышает доверие к полученному
  результату.
\item[Диалоговая система]  ориентирована на организацию дружественного
  общения с конечным пользователем как на этапе приобретения знаний,
  так и на этапе получения и объяснения результатов (решений).
\end{description}

Описанная система является статической, поскольку используется в тех
приложениях, где можно не учитывать изменения окружающего мира в
процессе получения решения.

\subsection{ЭС динамического типа}

В архитектуру динамической ЭС по сравнению со статической вводятся
дополнительно 2 компонента:
\begin{enumerate}
\item[1)] подсистема моделирования внешнего мира;
\item[2)] подсистема сопряжения с внешней средой.
\end{enumerate}

Подсистема сопряжения с внешним миром осуществляет связь в внешним
миром через набор датчиков и контроллеров.

Традиционные компоненты статических ЭС в динамических ЭС, прежде
всего БЗ и решатель, претерпевают значительные изменения, чтобы
отразить временную логику происходящих в реальном мире событий.

%===============================================================================

\section{Интеллектуальные информационные технологии и развитие
  аппарата знаний}

До последнего времени состояние исследований в развитии аппарата
знаний соответствовало следующей схеме:

Основные направления развития аппарата знаний:
\begin{enumerate}
\item Извлечение знаний из различных источников (формализация и
  интерпретация знаний.
\item Приобретение знаний от профессионалов.
\item Представление знаний:
  \begin{enumerate}
  \item Модели знаний:
    \begin{enumerate}
    \item Семантические сети.
    \item Фреймы (сети фреймов).
    \item Логические системы.
    \item Продукции.
    \end{enumerate}
  \item Системы представления знаний.
  \item Базы знаний.
  \end{enumerate}
\item Манипулирование знаниями.
\item Объяснение знаний.
\end{enumerate}

%===============================================================================

\section{Модели представления знаний (МПЗ)}

Типы МПЗ:
\begin{itemize}
\item Декларативные:
  \begin{itemize}
  \item продукционные;
  \item редукционные;
  \item предикатные.
  \end{itemize}
\item Процедурные:
  \begin{itemize}
  \item PLANNER;
  \item CONNIVER;
  \item ПРИЗ.
  \end{itemize}
\item Специальные (комбинированные):
  \begin{itemize}
  \item семантические сети;
  \item сети фреймов;
  \item нечёткие МПЗ;
  \item реляционные МПЗ.
  \end{itemize}
\end{itemize}

%===============================================================================

\section{Аппарат знаний и его влияние на интеллектуализацию
  информационной технологии}

В настоящее время наблюдаются следующие основные тенденции названного
влияния:
\begin{enumerate}
\item \emph{Переход от классических вычислений (архитектура фон Неймана) к
  альтернативным способам организации вычислительного процесса.}

  В течение нескольких последних десятков лет постоянно велись
  исследования по переходу от архитектуры фон Неймана к иным способам,
  связанным, в основном, с исследованиями в области искусственного
  интеллекта и параллельного программирования для многопроцессорных
  систем. Эти исследования в последнее время реализуются на моделях,
  строящихся децентрализованными, асинхронными, максимально
  параллельными, управляемыми по данным в процессе вычисления.
\item \emph{Технология активных объектов.}

  Ключевой в перестройке информационных технологий последних лет
  явилась реализация подхода на основе объектно-ориентированного
  программирования (ООП), однако этот подход определил пока лишь
  фундамент будущей технологии, оставляя прежним алгоритмический
  характер управления процессом вычислений.

  Тем временем, направление развития по данным и дальнейшее развитие
  направлений на основе событий формируют следующие направления
  интеллектуализации информационных технологий:
  \begin{itemize}
  \item на основе автономных активных объектов, интегрирующих
    мультиагентную архитектуру;
  \item методы программирования в ограничениях;
  \item аппарат недоопределённых моделей.
  \end{itemize}
\item \emph{Приоритет модели, а не алгоритма
    (см. разд.~\ref{sec:models_and_algorithms})}.

  Известен прогноз, который предполагает, что через 10--15 лет
  алгоритм ожидает судьба ассемблера и машинных кодов и потеря
  сегодняшних ключевых позиций, а также места в сравнительно тонком
  базовом уровне информационных технологий будущего
\item \emph{Параллелизм.}

  В новых технологиях параллельность перестаёт быть проблемой, а
  становиться естественным свойством любой программной системы.
\end{enumerate}



%===============================================================================

\section{Смена парадигмы разработки интеллектуальных информационных систем}

В теории программирования существуют 2 противоположных и
взаимодополняющих друг друга понятия:
\begin{enumerate}
\item Императивное (алгоритмическое, командное) \textrightarrow{} понятие
  алгоритма;
\item Декларативное (непроцедурное, основанное на моделях)
  \textrightarrow{} данные.
\end{enumerate}

\subsection{История развития информационных технологий}

Под \emph{моделью} можно понимать определенное множество абстрактных объектов
(несколько множеств абстрактных объектов), различающихся условно
приписываемым им именам в совокупности с заданной системой отношений
между элементами этих множеств.

Под \emph{алгоритмом} можно понимать точное предписание
последовательности действий, необходимых для получения искомых результатов.

\begin{table}[ht]
  \centering

  \topcaption{Этапы развития информационных технологий}
  \label{tab:stages}

  \begin{tabular}{|*{6}{m{.14\linewidth}|}}
    % \begin{tabular}{|p{2.2cm}|p{2.2cm}|p{2.2cm}|p{2.2cm}|p{2.2cm}|p{2.2cm}|}
    \multicolumn{3}{c|}{Модели} &
    \multicolumn{3}{c}{Алгоритмы}\\
    \hline
    \multicolumn{6}{c}{Этап 1:}\\ \cline{4-6}
    \multicolumn{3}{c|}{} & Раздел данных & Операторы обработки & Операторы управления\\ \cline{4-6}
    \multicolumn{6}{c}{Этап 2:}\\ 
    \cline{2-5}
    \multicolumn{1}{c|}{}&Данные & Модели данных & Операторы обработки & Операторы
    управления\\ \cline{2-5}
    \multicolumn{6}{c}{Этап 3:}\\ 
    \cline{1-4} Данные & Модели данных & Модели знаний & Опе\-ра\-ци\-он\-ная среда
    (решатель)\\ \cline{1-4}
  \end{tabular}
\end{table}


Таким образом, если на 1-ом этапе развития информационных технологий
раздел данных можно рассматривать как неразвитый элемент в большом
теле алгоритма, то интеллектуальный решатель на 3-ем этапе
представляет собой небольшой автономный элемент алгоритма в теле моделей.

%-------------------------------------------------------------------------------

\subsection{\label{sec:models_and_algorithms}Смена влияния моделей и алгоритмов на развитие
  информационных технологий}

В подтверждение сказанному выше, можно отметить следующее: на
начальном этапе развития вычислительной техники программирование в
кодах считалось единственно возможным способом общения с ЭВМ, однако
уже к настоящему времени этот тип программирования сохранил своё место
только в тонком, ближайшем к аппаратной части слое операционных
систем. По аналогии и модель завоёвывает своё место в практике
программирования. Именно модель представляет собой объект исследования
и определяет характер формального аппарата, используемого для описания
задач.

С моделью работает конечный пользователь --- специалист конкретной
предметной области. С алгоритмом же работает профессиональный
программист и разработчик вычислительных систем. Всё это определило
парадоксальность современного состояния прикладных информационных
технологий, когда модель можно встретить в основном только в
теоретических работах как иллюстрацию к объекту исследования.

%  \begin{tabular}{m{.45\linewidth}|m{.45\linewidth}}

\begin{table}[ht]
  \centering

  \topcaption{Качественное сопоставление модели и алгоритма}
  \label{tab:model_and_algorithm}

  \begin{tabular}{m{.45\linewidth}|m{.45\linewidth}} \hline
    \multicolumn{1}{c|}{\large\textbf{Модель}} &
    \multicolumn{1}{c}{\large\textbf{Алгоритм} \rule{0pt}{16pt}} \\[5pt]
    \hline \hline
    
    Принципиально декларативна & Антидекларативна\\ \hline
    
    Симметрична по отношению к своим параметрам, поскольку все они
    определяются друг через друга & Разделяет параметры на входные и
    выходные явным образом, определяя вторые через первые\\ \hline
    
    В неявной форме определяет решение всех задач, связанных с объектом
    исследования (программирования) & Определяет в явной форме и задаёт
    решение только одной задачи, отношение которой к реальному объекту не
    всегда очевидно\\ \hline
    
    Может быть недоопределённой & Алгоритм и недоопределённость "---
    несовместимые понятия\\ \hline
    
    В общем случае определяет всё пространство решений & Позволяет
    получать только отдельные точечные решения (в общем случае).\\ \hline
  \end{tabular}
\end{table}



Можно представить себе такую идеальную ситуацию, когда компьютер
способен взаимодействовать с конечным пользователем непосредственно
(напрямую). Получая на вход формальную модель или её описание,
компьютер автоматически <<сжимает>> её до минимального $k$-мерного
параллелепипеда в соответствующем гиперпространстве параметров
модели. При введении дополнительных ограничений или изменении
параметров модели этот параллелепипед в общем случае <<сжимается>> или
изменяет свои размеры (уточняется, дополняется), возможно исчезая
совсем, в том случае, если модель или введённые ограничения
несовместимы.

Такой идеальный способ организации вычислений на основе моделей в
форме сжатия пространства моделей был бы внутренне параллельным, а
также недетерминированным и асинхронным, и, следовательно,
естественным образом переносимым на параллельные ЭВМ.

\begin{rem}
  Очевидно, что для человека, сформировавшегося как программист или
  как конечный пользователь на основе императивной технологии,
  описанный подход представляется совершенно фантастическим и
  невозможным в принципе. Однако, такие технологии уже существуют, в
  частности, такие технологии обеспечивают пользователя почти всем
  спектром необходимых возможностей и позволяют работать с такими
  моделями без посредников, допуская в рамках одной модели сочетания
  различных формальных аппаратов (алгебры логики, функционального
  анализа, теории множеств) "--- принцип полимодельности.
\end{rem}

Подытожив сказанное, можно отметить, что на самом деле модель и
возможность прямого взаимодействия с нею являлись с самого начала
развития искусственного интеллекта ключевым ориентиром и естественным
следствием этих исследований была принципиальная потребность выхода за
пределы парадигмы алгоритма в самых различных направлениях:
\begin{itemize}
\item lisp;
\item prolog;
\item фреймы;
\item продукционные системы;
\item мультиагентные системы;
\item методы удовлетворения ограничений (недоопределённость
  вычислений).
\end{itemize}

Объективный прогноз развития информационных технологий говорит, что в
ближайшие 10--20лет алгоритм ожидает судьба ассемблера и
программирования в годах, потеря сегодняшних ключевых позиций и места
в сравнительно тонком базовом уровне будущей информационной технологии.

%%% Local Variables: 
%%% mode: latex
%%% TeX-master: "lections"
%%% End: 