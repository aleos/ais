\chapter{Системы искусственного интеллекта и основные принципы их построения}

\section{Этапы развития систем искусственного интеллекта}
Появление систем искусственного интеллекта (СИИ) и начало исследований
в области СИИ началось с конца 50-х годов с появлением работ Ньюэла,
Шоу.

3 этапа развития СИИ:
\begin{description}
\item[I этап:] середина 50-х--середина 60-х;
\item[II этап:] середина 60-х--середина 70-х;
\item[III этап:] с середины 70-х и до наших дней.
\end{description}

%-------------------------------------------------------------------------------
\subsection{Характеристика I этапа развития CИИ}
I этап: Разработка и создание СИИ, решающих задачи на основе
эвристических методов.

Эвристический метод "--- свойственный человеческому мышлению метод, для
которого характерно использование догадок о путях решения задач с
последующей их проверкой.

Противоположность <<алгоритму>> "--- <<эвристика>>.

Эвристические методы используются когда нельзя установить жёсткие
условия. Первоначально широкое распространение получили разработки,
моделирующие мыслительную деятельность человека.

Область применения СИИ (I этап):
\begin{itemize}
\item игры (разбиение задачи на классы, моделирование на 2-3 шага
  вперёд);
\item головоломки;
\item математические задачи познавательного характера;
\item и др.
\end{itemize}

Эти области применения характеризовались простотой и ясностью
предметной области (проблемной среды), её относительно малой
громоздкостью, возможностью прямого перебора или возможностью подбора
под модель.

%-------------------------------------------------------------------------------
\subsection{Характеристика II этапа развития CИИ}
II этап: Разработка и создание СИИ, ориентированных на применение
интегральных роботов.

При переходе на I этапе от искусственных предметных областей (ПрО) к
реальным разработчики стали сталкиваться с большими трудностями,
обусловленными необходимостью моделирования реального мира:
\begin{itemize}
\item описание знаний о внешнем мире;
\item организация хранения этих знаний;
\item эффективный поиск и доступ к знаниям;
\item проверка знаний на корректность (полнота, непротиворечивость,
  верифицируемость и пр.).
\end{itemize}

Практическое применение СИИ привело к созданию <<интегральных
роботов>>, которые привели к необходимости рассмотрения и
использования реальных ПрО.

Переход к реальным ПрО обусловлен переходом от исследований по
моделированию способов мышления человека к разработке программ,
способных решать <<человеческие задачи>>, но формальным методом
(математическая логика, смысловая логика). К этому периоду относится
разработка метода автоматического доказательства теорем на основе
метода резолюций (1969 г., Робинсон, Минский, Пайперт).

Для СИИ II этапа характерно:
\begin{itemize}
\item наличие формальной модели предметной области;
\item наличие алгоритма распознавания ситуаций, изображений, сцен;
\item алгоритм принятия решений;
\item алгоритмы планирования работы интегральных роботов;
\item алгоритм оценки качества выполненных/спланированных работ;
\item и др.
\end{itemize}

Появление первых экспериментальных образцов роботов, решающих свои
задачи в комплексе показало необходимость решения фундаментальных
проблем, связанных со следующими задачами:
\begin{itemize}
\item представление знаний;
\item зрительное восприятие (распознавание образов);
\item построение сложных планов поведения в динамических средах;
\item общение с роботами на естественном языке;
\item и др.
\end{itemize}

Всё это привело к III этапу развития СИИ.

%-------------------------------------------------------------------------------
\subsection{Характеристика III этапа развития СИИ}
III этап: Разработка и создание СИИ, характерной чертой которых
явился переход от создания автономно функционирующих систем,
самостоятельно решающих задачу в реальной среде, к созданию
человеко-машинных систем, совмещающих достоинства интеллекта человека
и возможности ЭВМ для достижения общей цели решения задачи,
поставленной перед интегральной человеко-машинной решающей системой.

Причины возникновения человеко-машинных систем:
\begin{enumerate}
\item к этому времени выяснилось, что даже простые задачи
  функционирования интегрального робота в реальных ПрО не могут быть
  решены методами, разработанными для роботов в экспериментальных ПрО;
\item Стало ясно, что сочетание дополняющих друг друга способностей
  человека и возможностей ЭВМ позволяет обойти нерешаемые задачи
  (острые углы) путём перекладывания на человека тех функций, которые
  пока недоступны для ЭВМ.
\end{enumerate} 

\begin{table}[ht]
  \centering
  \begin{tabular}{p{.35\linewidth}|@{\qquad}p{.35\linewidth}}
    \textbf{Способности человека:}
    \begin{itemize}
    \item интуиция;
    \item эвристика.
    \end{itemize}
   & 
    \textbf{Способности ЭВМ:}
     \begin{itemize}
    \item скорость;
    \item хранение данных;
    \item надёжность.
    \end{itemize}
  
    \end{tabular}
\end{table}

СИИ "--- система, в которой:
\begin{itemize}
\item развиваются возможности ЭВМ в направлении обеспечения
  совместного с пользователем решения задач;
\item упрощаются процессы общения человека и ЭВМ в ходе решения задач;
\item постоянно расширяется доля компьютеров в совместной с человеком
  деятельности по решению задач;
\item значительное внимание уделяется повышению способности ЭВМ к
  самостоятельному решению (в автоматическом режиме) трудно решаемых
  задач.
\end{itemize}

%===============================================================================
\section{Новая информационная система обработки информации}

\subsection{Основные черты традиционной (старой) обработки данных}
\begin{itemize}
\item Наличие следующих этапов обработки информации (рис.~\ref{fig:traditional_data_processing}):
  \begin{enumerate}
  \item Специалист (эксперт) "--- конечный пользователь.
  \item Системный аналитик.
  \item Программист.
  \item Вычислительная система.
  \end{enumerate}

  \begin{figure}[ht]
    \centering
      %\fboxsep=0pt \fboxrule=1pt
      \begin{picture}(450,160)
        %% \framebox — бокс заданного размера
        %% \dashbox — бокс заданного размера с пунктирной границей
        %% \parbox — абзац текста заданного размера
        %% \makebox — строка текста заданной длины
        \thicklines % «толстые» линии
        %\put(0,0){\dashbox{5}(450,160){}}
        % Блоки с надписями (этапы)
        \put(0,75){\framebox(90,50){\parbox[c][50pt]{90pt}{\centering
              Специалист\par
              Эксперт\par
              (КП)\footnotemark}}}
        \put(120,75){\framebox(90,50){\parbox[c][50pt]{90pt}{\centering
              Системный
              аналитик}}}
        \put(240,75){\framebox(90,50){\parbox[c][50pt]{90pt}{\centering
              Программист}}}
        \put(360,75){\framebox(90,50){\parbox[c][50pt]{90pt}{\centering
              Вычисли\-тельная
              система}}}
        % Стрелки между блоками в обе стороны
        \put(90,110){\vector(1,0){30}}
        \put(120,90){\vector(-1,0){30}}
        \put(210,110){\vector(1,0){30}}
        \put(240,90){\vector(-1,0){30}}
        \put(330,110){\vector(1,0){30}}
        \put(360,90){\vector(-1,0){30}}
        % Кривая стрелка сверху с надписью
        \qbezier(170,125.5)(225,155)(280,125.5)
        \put(280,125.6){\vector(2,-1){0}}
        \put(225,145){\makebox[0pt]{\small{Алгоритм, входные данные}}}
        % Скруглённый прямоугольник снизу с надписью
        \put(225,20){\oval(350,40)}
        \put(50,16){\makebox[350pt]{
            Результаты решения задачи на каждом этапе решения}}
        % Стрелки от нижнего скруглённого прямоугольника
        \put(105,40){\vector(0,1){50}}
        \put(225,40){\vector(0,1){50}}
        \put(345,40){\vector(0,1){50}}
      \end{picture}

    \caption{Этапы обработки информации при традиционной (старой) обработке данных}
    \label{fig:traditional_data_processing}
  \end{figure}

  \footnotetext{Конечный пользователь}

\item Большая трудоёмкость решения конкретной задачи обработки
  информации, связанная с имеющимися место на каждом этапе:
  \begin{itemize}
    \item ошибками;
    \item неточностями;
    \item нерациональными решениями;
    \item необходимостью внесения изменений в схемы решения, методы
      решения, постановку задачи, программу.
  \end{itemize}
\item Сложность сопровождения разработанного программного обеспечения
  (ПО).
\item ПО в традиционной технологии основывается на формальной
  (математической) модели задач и представления данных, в то время как
  каждая конкретная предметная область основывается на системе,
  содержательных понятий, которыми оперирует пользователь при решении
  задач. 

  Это значит, что в такой схеме решения система понятий ПрО и
  формальной модели не совпадают.
\item При формулировке задачи конечный пользователь должен перевести
  постановку задачи, выраженной в системе ПрО в постановку задачи
  формальной модели (рис.~\ref{fig:interpretation}), при получении
  результата "--- наоборот (рис.~\ref{fig:opposite_interpretation}).
\end{itemize}

\begin{figure}[ht]
  \centering
  \begin{picture}(450,75)
    \thicklines % «толстые» линии
    % \put(0,0){\dashbox{5}(450,75){}}
    
    \put(55,50){\oval(110,50)}
    \put(0,47){\parbox[c][50pt]{110pt}{\centering
        Формулировка
        задачи}}
    
    \put(132,46){\bfseries \Large :}

    \put(160,25){\framebox(110,50){\parbox[c][50pt]{110pt}{\centering
          Содержательная
          постановка
          задачи}}}

    \put(270,50){\vector(1,0){70}}
    \put(300,34){\bfseries \large *}

    \put(340,25){\framebox(110,50){\parbox[c][50pt]{110pt}{\centering
          Формальная
          постановка
          задачи}}}

    \put(10,0){{\large *} "--- интерпретация}
  \end{picture}
  \caption{Процесс интерпретации постановки задачи}
  \label{fig:interpretation}
\end{figure}

\begin{figure}[ht]
  \centering
  \begin{picture}(450,75)
    \thicklines % «толстые» линии
    % \put(0,0){\dashbox{5}(450,75){}}
    
    \put(55,50){\oval(110,50)}
    \put(0,47){\parbox[c][50pt]{110pt}{\centering
        Оценка\\
        результатов}}
    
    \put(132,46){\bfseries \Large :}

    \put(160,25){\framebox(110,50){\parbox[c][50pt]{110pt}{\centering
          Формальные результаты}}}

    \put(270,50){\vector(1,0){70}}
    \put(297,34){\bfseries \large **}

    \put(340,25){\framebox(110,50){\parbox[c][50pt]{110pt}{\centering
          Содержательные
          результаты}}}

    \put(10,0){{\large **} "--- обратная интерпретация}
  \end{picture}
  \caption{Процесс интерпретации полученного результата решения задачи}
  \label{fig:opposite_interpretation}
\end{figure}

%-------------------------------------------------------------------------------
\subsection{Основная идея новой технологии обработки данных}
Система понятий конкретной предметной области рассматривается как
исходная информация для решения прикладных задач, при этом
обеспечивается автоматическая интерпретация системы понятий формальной
модели.

Для новой технологии характерна следующая цепочка обработки данных (рис.~\ref{fig:new_data_processing}):
\begin{enumerate}
\item Специалист (эксперт) "--- конечный пользователь;
\item Содержательная постановка задачи;
\item Вычислительная система.
\end{enumerate}

\begin{figure}[ht]
  \centering
  \begin{picture}(400,85)
    \thicklines % «толстые» линии
    %\put(0,0){\dashbox{1}(400,85){}}
    
    \put(0,35){\framebox(100,50){\parbox[c][50pt]{100pt}{\centering
          Специалист\\
          Эксперт\\
          (КП)}}}

    \put(100,65){\parbox[c][50pt]{200pt}{\centering\small
        Содержательная постановка задачи}}
    
    \put(100,60){\vector(1,0){200}}

    \put(300,35){\framebox(100,50){\parbox[c][50pt]{100pt}{\centering
          Вычислительная
          система
          (ВС)}}}

    \qbezier(83,33)(200,-7)(320,34.5)
    \put(80,34.85){\vector(-2,1){0}}

    \put(200,2){\makebox[0pt]{\small Содержательные результаты}}

  \end{picture}
  \caption{Цепочка обработки данных в новой технологии}
  \label{fig:new_data_processing}
\end{figure}

В новой технологии обработки информации вычислительной системе должна
задаваться только постановка задачи в виде описания требуемого
результата и условий его получения, в то время как последовательность
операций, посредством которых она решается, определяется решающей
задачу системой.

Вычислительная система в составе человеко-машинной системы
превращается из пассивного звена в активную целенаправленную систему,
целью которой является получение требуемого в постановке задачи
результата или знания.

Получение этого знания достигается:
\begin{itemize}
\item автоматическим синтезом (генерацией) вычислительной системы;
\item выполнением оптимальной в определённом смысле последовательности
  вычислительных, логических и поисковых операций над имеющимися знаниями.
\end{itemize}

%-------------------------------------------------------------------------------
\subsection{Структура ВС в новой технологии обработки информации}
Основные структурные элементы вычислительной системы (ВС) (см. рис.~\ref{fig:new_computing_system}):
\begin{enumerate}
\item Модель языка пользователя.
\item Модель представления знаний.
\item База знаний.
\item Интеллектуальный интерфейс:
  \begin{itemize}
  \item система общения;
  \item решатель.
  \end{itemize}
\item Исполнительная система.
\end{enumerate}

\begin{figure}[htb]
  \centering
  \begin{picture}(450,200)
    \thicklines % «толстые» линии
    %\put(0,0){\dashbox{1}(450,200){}}
    
    \put(95,100){\dashbox{7}(300,100){}}
    \put(105,140){\framebox(100,50){\parbox[c][50pt]{100pt}{\centering
          Система\\
          общения}}}
    \put(205,165){\vector(1,0){80}}
    \put(205,165){\vector(-1,0){0}}
    \put(285,140){\framebox(100,50){\parbox[c][50pt]{100pt}{\centering
          Решатель}}}
    \put(385,165){\vector(1,0){45}}
    \put(385,165){\vector(-1,0){0}}
    \put(245,105){\makebox[0pt]{Интеллектуальный интерфейс}}

    \put(430,0){\framebox(20,200){\parbox[c][200pt]{20pt}{\centering\rotatebox{90}{
            Исполнительная система}}}}

    \put(0,150){\parbox[c]{75pt}{\centering
        Модель\\
        языка\\
        пользователя}}
    \put(37.5,125){\vector(0,-1){50}}
    \put(37.5,125){\vector(0,1){0}}
    \put(0,45){\parbox[c]{75pt}{\centering
        Модель\\
        представления\\
        знаний}}
    
   %  {
   %   \thinlines
   %   % Верхний ромб для рисования верхней поверхности
   %   \put(150,60){\line(4,1){95}}
   %   \put(340,60){\line(-4,1){95}}
   %   \put(150,60){\line(4,-1){95}}
   %   \put(340,60){\line(-4,-1){95}}
   %   % Половина нижнего ромба для рисования нижней поверхности
   %   \put(150,30){\line(4,-1){95}}
   %   \put(340,30){\line(-4,-1){95}}
   % }
   % % Вспомогательные точки верхней поверхности:
   % \put(197.5,71.875){\circle*{5}} % левая точка верхней дуги
   % \put(292.5,71.875){\circle*{5}} % правая точка верхней дуги

   % \put(197.5,48.175){\circle*{5}} % левая точка нижней дуги
   % \put(292.5,48.175){\circle*{5}} % правая точка верхней дуги

   % \put(245,83.75){\circle*{5}} % верхний угол ромба
   % \put(245,36.25){\circle*{5}} % нижний угол ромба

   % % Вспомогательные точки нижней поверхности
   % \put(197.5,18.175){\circle*{5}} % левая точка нижней дуги
   % \put(292.5,18.175){\circle*{5}} % правая точка нижней дуги
   
   % \put(173.75,24){\circle*{2}} % опорная левой дуги
   % \put(316.25,24){\circle*{2}} % опорная правой дуги

   % Верхняя поверхность:
   \qbezier(197.5,71.875)(245,83.75)(292.5,71.875) % верхняя дуга
   \qbezier(197.5,48.175)(245,36.25)(292.5,48.175) % нижняя дуга
   \qbezier(197.5,71.875)(150,60)(197.5,48.175) % левая дуга
   \qbezier(292.5,71.875)(340,60)(292.5,48.175) % правая дуга

   % Боковые грани
   \put(173.75,30){\line(0,1){30}}
   \put(316.25,30){\line(0,1){30}}
 
   % Нижняя поверхность:
   \qbezier(197.5,18.175)(245,6.25)(292.5,18.175) % нижняя дуга
   \qbezier(197.5,18.175)(173.75,24)(173.75,30) % нижняя левая дуга
   \qbezier(292.5,18.175)(316.25,24)(316.25,30) % нижняя правая дуга

   % Подпись «БЗ»
   \put(245,20){\makebox[0pt]{\Large БЗ}}
 
   % Левая верхняя стрелка от БЗ
   \put(197.5,71.875){\line(-3,1){50}}
   \put(147.5,88.541667){\vector(0,1){51,45833}}
   % Правая верхняя стрелка от БЗ
   \put(292.5,71.875){\line(3,1){50}}
   \put(342.5,88.541667){\vector(0,1){51,45833}}
   % Левая стрелка
   \put(85,45){\vector(1,0){88.75}}
   % Правая стрелка
   \put(316.25,45){\vector(1,0){113}}
   \put(316.25,45){\vector(-1,0){0}}
      
  \end{picture}
  \caption{Цепочка обработки данных в новой технологии}
  \label{fig:new_computing_system}
\end{figure}

Система программных и аппаратных средств, обеспечивающих для конечного
пользователя (КП), не имеющего специальной подготовки в области
вычислительной техники использования ЭВМ для решения задач,
возникающих в сфере профессиональной деятельности либо без
посредников-программистов либо с незначительной их помощью, называется
\emph{интеллектуальным интерфейсом}. 

Процесс же внедрения средств интеллектуального интерфейса в
вычислительную технику называется интеллектуализацией ЭВМ.  Под
термином <<\emph{интеллектуализация}>> понимается (подчёркивается) то,
что с одной стороны имеется нетривиальность функций, выполняемых этим
интеллектуальным интерфейсом, и с другой стороны то, что выполнение
этих функций до последнего времени является прерогативой человека.

\emph{Система общения} "--- совокупность средств, осуществляющих трансляцию с
языка пользователя (конечного пользователя) на язык представления
знаний в базе знаний (БЗ) и включающую в себя средства трансляции и
средства, обеспечивающие понимание.

\emph{Решатель} "--- совокупность средств, обеспечивающих в диалоге с
пользователем автоматический синтез программы решения задач,
т.е. анализ условий задачи, выделение подзадач, имеющих стандартные
решения, и объединение этих подзадач в единую целостную
программу. Функционирование системы, имеющей в своем составе решатель,
может осуществляться в 2-х режимах: в режиме компиляции и режиме
интерпретации.

\emph{Исполнительная система} представляет собой совокупность средств,
выполняющих (исполняющих) синтезированную программу вычислений,
сформированную с позиции эффективного решения задач и имеет проблемную
ориентацию.

\emph{База знаний} занимает центральное положение по отношению к остальным
компонентам вычислительной системы. Через БЗ осуществляется интеграция
всех средств вычислительной системы, участвующих в решении конкретных
задач.

Базу знаний для вычислительных систем, ориентированных на
использование новой технологии обработки данных целесообразно строить
как 2-уровневую структуру, включающую 2 компонента:
\begin{enumerate}
\item концептуальную БЗ (верхний концептуальный уровень);
\item базы данных (БД) (нижний информационный уровень).
\end{enumerate}
Такое деление позволяет обеспечить эффективное представление
обобщенных знаний и метазнаний на верхнем концептуальном уровне и
конкретной информации (данных) на нижнем уровне.

Использование баз данных в качестве отдельного компонента баз знаний
позволяет использовать современные СУБД для описания взаимодействий
(взаимоотношений) между объектами проблемной среды, причем на уровне
конкретных фактов.

\emph{Язык представления знаний} "--- это конкретный способ описания знаний в
проблемной среде, задаваемый синтаксисом описания и правилами
соотнесения языковых выражений с конкретными объектами проблемной
области или интерпретацией.

% ===============================================================================

% ===============================================================================
% ===============================================================================

%%% Local Variables: 
%%% mode: latex
%%% TeX-master: "lections"
%%% End: 
