\appendix
\chapter{Программный комплекс мониторинга состояний и управления
  сложными техническими объектами в реальном масштабе времени} 
\section{Назначение, характеристики и структура информационных систем
  мониторинга состояния сложных технических объектов}

В настоящее время состояние дел в области проектирования и
эксплуатации программных комплексов мониторинга состояния сложных
технических объектов может быть охарактеризовано рядом положений:
\begin{itemize}
\item большие (сверхбольшие) потоки обрабатываемой информации;
\item возрастание сложности объектов управления;
\item увеличение количества объектов управления;
\item большое разнообразие типов измерительной информации,
  используемой для принятия решений;
\item неопределённость и нечётность задач при проведении мониторинга
  задач сложных технических  объектов.
\end{itemize}

%===============================================================================

\section{Характеристика основных элементов информационной технологии,
  которую реализует программный комплекс}

\subsection{Назначение информационной технологии, реализуемой
  программным комплексом}

Мониторинг состояний (МС) предполагает получение в явном виде
обобщённых оценок выполнения программы функционирования
рассматриваемого объекта управления, либо степень его
работоспособности, места и вида возникшей неисправности, оценок
прогнозируемых процессов с заданной точностью, с учётом конкретных
целей и условий эксплуатации на различных этапах его функционирования.

Мониторинг может проводиться при интеграции всех имеющихся видов
измерительной информации и решать следующий перечень задач:
\begin{itemize}
\item \emph{контроль функционирования объекта управления},
  выполняющийся при его нахождении как штатных, так и внештатных
  ситуациях; при реализации технологии автоматизированного управления
  при мониторинге может быть реализован режим выдачи команд оператора
  с АРМ;
\item \emph{контроль работоспособности объекта управления} и, при
  возникновении неисправности, её диагностирование с указанием места и
  вида возникшей ситуации;
\item \emph{прогнозирование поведения объекта управления} и
  предсказание развития как штатных, так и нештатных (аварийных)
  ситуаций с целью их предупреждения и недопущения.
\end{itemize}

Основным элементом автоматизации процесса МС является функциональный
элемент этой системы, решающий задачу по сбору, обработке и анализу
всех видов измерительной информации для произвольного типа и
количества пользователей.

Таким функциональным элементом является унифицированный типовой модуль
автоматизации, объединяющий в себе систему поддержки интерфейса
че\-ло\-век-ма\-ши\-на и программное обеспечение операторских станций.

%===============================================================================
\subsection{Характеристика унифицированного типового модуля
  автоматизации (УТМА)}

При создании автоматизированных систем управления технологическими
процессами любой сложности всегда существовала тяжело решаемая
проблема: каким образом реализовать технологию формализации знаний о
функционировании объектов управления (ОУ). Неоднократно
предпринимались попытки реализовать такие технологии путём
использования специализированных языков (блок-схемы, циклограммы
функционирования и т.д.), однако все эти попытки были обречены на
провал. Причиной послужило то, что очень трудно (почти невозможно)
добиться от конструктора ОУ хотя бы словесного описания алгоритма
манипулирования данными при МС, и дажи при получении такого описания,
переложить его на язык программирования. Основная причина такого
положения --- принципиально разные языки, на которых общаются
программисты и технологи: программисты используют языки
программирования, технологи --- технологические языки описания
соответствующей предметной области.

Единственный выход из этой ситуации --- предоставить конечному
пользователю понятное ему средство для описания соответствующей
предметной области, объяснив при этом возможность самостоятельно (без
участия программиста) описывать технологические процессы
функционирования ОУ. Основное достоинство рассматриваемого УТМА
состоит в том, что с его помощью создаются (<<программируются>>)
уникальные модули автоматизации, причём самими технологами, почти без
участия программистов. Вся работа технолога состоит в том, что он
отлаживает непосредственно сам процесс технологического управления,
записанного на понятном пользователю языке и состоящем из примитивов,
которыми этот конечный пользователь оперирует. При этом технологу
необходимы такие основные примитивы для описания своей предметной
области, которые включают типовой набор функций, присущий всем
процессам автоматизации для мониторинга состояния соответствующих ОУ:
\begin{itemize}
\item экранные формы отображения значений, измеряемых параметров типа
  стрелочных, полосковых или цифровых индикаторов, а также
  сигнализирующие панели различной формы и содержания, например,
  индикатор температуры;
\item возможность создания архивов штатных и нештатных ситуаций,
  событий и поведения динамических процессов во времени (так
  называемые <<тренды>>), а также полное и выборочное сохранение
  измеряемых параметров по времени и по условию;
\item упрощенный, адаптированный к предметной области язык для
  реализации алгоритмов обработки информации, математических и
  логических выражений (например, язык схем анализа, используемый на
  лабораторных работах;
\item ядро (монитор реального времени), которое обеспечивает
  детерминизм поведения информационной системы или, иными словами,
  предскозуемое время отклика на внешнее событие. Существует понятие
  системы рельного времени. В такой системе жёстко задано максимальное
  время отклика, равное $\Delta t$;
\item драйверы управления различного рода внешним оборудованием;
\item средства, реализующие сетевые функции;
\item средства защиты от несанкционированного доступа;
\item многооконный графический интерфейс и другие очевидные функции,
  такие как импорт изображений, создание собственной библиотеки
  алгоритмов, динамических объектов и др.;
\item средства ведения протокола работы пользователя при проведении
  сеансов работы с данными;
\item средства обеспечения работы исполнительной системы на
  разнородных про\-грамм\-но-ап\-па\-рат\-ных платформах;
\item и др.
\end{itemize}

При обеспечении функционирования информационной системы этапность
работы конечного пользователя определяется следующей схемой:
\begin{enumerate}
\item Формирование статического изображения рабочего стола (окна):
  \begin{itemize}
  \item мнемосхема контролируемого объекта;
  \item информационные и управленческие связи на мнемосхеме;
  \item органы управления различных типов (кнопки, выключатели и др.);
  \item и др.
  \end{itemize}
\item Формирование динамических объектов рабочего стола:
  \begin{itemize}
  \item стрелочные индикаторы;
  \item полосковые индикаторы;
  \item цифровые индикаторы;
  \item тренды контролируемых процессов;
  \item произвольные сигнализирующие табло;
  \item и др.
  \end{itemize}
\item Описание взаимодействия между имеющимися этапами на мнемосхемах:
  \begin{itemize}
  \item динамические объекты взаимодействия;
  \item входные/выходные каналы измерительной и обработанной
    информации;
  \item и др.
  \end{itemize}
\end{enumerate}

%===============================================================================
\section{Режимы распределенной иерархической обработки данных}

В распределенных иерархических системах сбора и обработки данных
выделяются несколько уровней:
\begin{itemize}
\item уровень непосредственного сбора данных. Основан на использовании
  датчиков, регуляторов и исполнительных механизмов, на которых
  программное обеспечение загружается с ПЗУ в ОЗУ, флеш-память и др.
\item основной уровень. На нём собирается вся необходимая информация
  от источников низшего уровня и включает в свой состав не только
  вычислительные средства, но и человека-оператора.
\item уровень оценки состояния ОУ. На этом уровне выполняются операции
  по модернизированию, прогнозированию, оптимизации процессов, и на
  него поступает информация, собранная на основном уровне. При этом
  предполагается использование мощных вычислительных ресурсов. Модуль
  автоматизации на этом уровне строится на базе экспертных,
  расчётно-логических или моделирующих систем обработки данных
  рельного времени.
\end{itemize}

Типовой модуль автоматизации на любом из отмеченных уровней состоит
из:
\begin{enumerate}
\item Базы данных измерений (БДИ) и диалогового редактора баз данных.
\item Графического редактора статических и динамических изображений на
  мнемосхемах.
\item Графического редактора символов, который позволяет создавать
  библиотеки типовых пиктограмм, используемых графическими редакторами
  статических и динамических изображений.
\item Средств сбора и отображения данных в предыстории по любому
  параметру из базы данных.
\item Генератора отчётов, который позволяет формировать отчеты по
  данным рельного времени и по предыстории.
\item Средств отображения событий реального времени (циклограммы,
  мнемосхемы, тренды, свотки событий и тревог, звуковая и речевая
  сигнализация и др.).
\item Средств обработки значений параметров и вычислений, задаваемых
  пользователем по алгоритму.
\end{enumerate}

Распределённые системы информационного обеспечения включают следующие
компоненты:
\begin{itemize}
\item средства поддержки сетевой работы в рамках распределенной
  системы, как в составе ЛВС различных технологий, так и в составе
  ГВС;
\item средство обмена данными и сообщениями между операторами и
  рабочими станциями;
\item средства пароля и защиты, разграничения доступа по уровню прав
  оператора;
\item система <<горячего>> резервирования и автоматического
  восстановления для обеспечения надежности, устойчивости и
  непрерывности вычислительного процесса.
\end{itemize}

%===============================================================================
\section{Основные технологических элементы информационной технологии
  обработки данных}

Основным технологическим элементом рассматриваемой технологии является
программный комплекс, реализующий всю её функциональность. Структура
программного комплекса:
\begin{itemize}
\item операционная среда (система автоматизированной подготовки
  исходных данных и знаний);
\item измерительная среда (система автоматизированной обработки
  (анализа) измерительной информации);
\item система генерации и обслуживания программного комплекса.
\end{itemize}

Структура операционной среды включает в себя следующие базовые
элементы:
\begin{itemize}
\item монитор операционной среды;
\item операционную систему;
\item сетевые средства;
\item средства компонентных вычислений (для платформы Windows ---
  DM-системы, для UNIX --- CORBA).
\end{itemize}

Структура функциональных подсистем:
\begin{enumerate}
\item Система баз данных.
\item Специализированные интерактивные подсистемы:
  \begin{itemize}
  \item концептуальное моделирование;
  \item поведенческое моделирование;
  \item GUI;
  \item автоматический синтез корректной метапрограммы.
  \end{itemize}
\item Языковые средства операционной среды:
  \begin{itemize}
  \item подсистема описания измеряемых и вычисляемых параметров;
  \item подсистема описания групп параметров для их совместной
    обработки;
  \item подсистема описания моделей сегментации значений параметров;
  \item подсистема описания динамических моделей изменения значений
    параметров;
  \item подсистема описания диалоговых панелей отображения;
  \item подсистема запросов к базе данных;
  \item подсистема контроля функционирования объекта управления;
  \item подсистема проверки условий и изменения ходя вычислений;
  \item подсистема вызова автоматизированных операций над данными;
  \item подсистема графического (мультимедийного) отображения;
  \item подсистема организации диалога с конечным пользователем.
  \end{itemize}
\end{enumerate}
