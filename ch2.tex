\chapter{Логические основы построения систем
искусственного интеллекта}

\section{Формальные теории (логические исчисления)}

Определение формальной теории (ФТ): любая формальная теория оперирует
с 2 объектами:
\begin{enumerate}
\item языком (определенным множеством высказываний, имеющих смысл с
точки зрения этой теории);
\item совокупностью теорем (подмножеством языка и состоящих из
высказываний, истинностных данной теории).
\end{enumerate}
Формальная система:
$$ FS = \left<A,A_1,R\right>\,,
$$
где $A$ "--- алфавит, $A_1$ "--- аксиомы, $R$ "--- правила вывода.

Формальная теория (исчисление) определяется следующим кортежем:
$$ FT=\left<U,L,S,R\right>\,,
$$
где $U$ "--- алфавит формальной теории, $L$ "--- язык формальной теории,
$S$ "--- аксиомы формальной теории, $R$ "--- правила вывода.


\begin{rem}[отличие формальной теории от формальной системы] Аксиомы и
теоремы формальной системы (ФС) трактуются как некоторые формулы или
предложения или как правильно построенные формулы (ППФ) без придания
им какого-либо смысла. Это означает, что в формальных системах
правильно построенные формулы сводятся к рассмотрению их структур или
синтаксических свойств соответствующих предложений.
\end{rem}

Рассмотрим более подробно элементы формальной теории:
\begin{itemize}
\item[$U$ "---] алфавит формальной теории:
  \begin{itemize}
  \item символы предметных констант $\{a,b,c,\ldots\}$;
  \item символы предметных переменных
$\{x,y,z,\ldots,x_1,y_1,z_1,\ldots\}$;
  \item символы функциональных констант
$\{f_1^{n1},f_2^{n2},f_3^{n3},\ldots\}$;
  \item символы предикатных констант $\{A,B,C,\ldots,X,Y,Z,\ldots\}$;
  \item символы предикатных переменных $\{\mathcal{A,B,C,\ldots}\}$;
  \item символы логических связок и кванторов
$\{\lor,\&,\lnot,\supset,\exists,\forall,\ldots\}$;
  \item вспомогательные символы $\{\;,\;,\;.\;,\;)\;,\;(\;\}$.
  \end{itemize}
\item[$L$ "---] язык формальной теории; определяется индивидуально, с
помощью различных конструктивных процедур и, как правило, рекурсивно.
\item[$S$ "---] аксиомы, $S\subseteq L$.

  Отличие формальной теории от формальных систем: каждой правильно
построенной формуле (ППФ) может быть сопоставлено некоторое
истинностное значение в каждой конкретной интерпретации.

  В формальной теории различают следующие виды ППФ:
  \begin{enumerate}
  \item \emph{Общезначимые} формулы, или тавтологии, "--- являются
тождественно истинностными во всех интерпретациях.
  \item \emph{Выполнимые} формулы "--- принимают истину или ложь в
зависимости от интерпретации.
  \item \emph{Противоречивые} "--- ложны во всех интерпретациях.
  \end{enumerate}

\item[$R$ "---] правила вывода.

  Обозначают $F_1,F_2,\ldots,F_n\vdash G$ или $\frac
{F_1,F_2,\ldots,F_n} G$, где $F_1,F_2,\ldots,F_n,G\in L$
  
  Выводом некоторой формулы $B$ из $A_1,A_2,\ldots,A_m$ называется
такая последовательность формул $F_1,F_2,\ldots,F_n$, что $F_m=B$, а
каждая $F_i$ есть:
  \begin{enumerate}
  \item Аксиома.
  \item Форма посылок ($F_i\in \{A_1,A_2,\ldots,A_m\}$).
  \item Выводима ($F_i\vdash F_1,F_2,\ldots,F_{i-1}$).
  \end{enumerate}

  Этот факт записывается $A_1,A_2,\ldots,A_m\vdash B$. В этом случае
$A_1,A_2,\ldots,A_m$ "--- посылки (гипотезы) вывода, $B$ "--- теорема
(вывод).

  Доказательством формулы $B$ в формальной теории называется вывод этой
формулы из пустого множества формул. В этом случае посылками являются
только аксиомы. В таком случае говорят, что формула $B$ выводима
(доказуема) в формальной теории.
\end{itemize}

%===============================================================================
\section{Исчисление высказываний}

Определение формальной теории исчисления высказываний:
$FT_{ИВ}=\left<U,L,S,R\right>$,
\begin{description}
\item[$U$]:
  \begin{itemize}
  \item переменные высказываний (пропозициональные символы):
$A,B,C.\ldots$;
  \item символы логических связок: $\lor,\&,\supset,\lnot$;
  \item дополнительные символы;
  \item иные способы задания $U$ нет.
  \end{itemize}
\item[$L$]:
  \begin{itemize}
  \item высказывательная переменная является ППФ;
  \item если $\mathcal A$ и $\mathcal B$ "--- ППФ, то такими формулами
являются также $\mathcal{A\lor B}$, $\mathcal{A\;\&\;B}$,
$\mathcal{\lnot B}$, \ldots
  \item других способов задания ППФ нет.
  \end{itemize}

  \begin{rem} Формулы, состоящие из единственного символа или
высказывательной переменной, называются \emph{атомарными}
(элементарными).
  \end{rem}
  
\item[$S$]:
  \begin{description}
  \item[$S_{I}$]:
    \begin{enumerate}
    \item $A\supset \left(B\supset A\right)$;
    \item $\left(A\supset
B\right)\supset\Bigl(\bigl(A\supset\left(B\supset
C\right)\bigr)\supset\left(A\supset C\right)\Bigr)$;
    \item $(A\with B)\supset A$;
    \item $(A\with B)\supset B$;
    \item $A\supset\bigl(B\supset (A\with B)\bigr)$;
    \item $A\supset(A \lor B)$;
    \item $B\supset(A \lor B)$;
    \item $\left(A\supset C\right)\supset\Bigl(\left(B\supset
C\right)\supset\bigl(\left(A\lor B\right)\supset C\bigr)\Bigr)$;
    \item $\left(A\supset B\right)\supset\bigl(\left(A\supset\lnot
B\right)\supset\lnot A\bigr)$;
    \item $\lnot\lnot A\supset A$.
    \end{enumerate}
  \item[$S_{II}$]:
    \begin{enumerate}
    \item $A\supset(B\supset A)$;
    \item $\bigl(A\supset\left(B\supset
C\right)\bigr)\supset\bigl(\left(A\supset
B\right)\supset\left(A\supset C\right)\bigr)$;
    \item $\left(\lnot A\supset\lnot B\right)\supset \bigl(\left(\lnot
A\supset B\right)\supset A\bigr)$;
    \item[*.] $A\lor B=\lnot A\supset B$;
    \item[*.] $A\with B=\lnot(A\supset\lnot B)$.
    \end{enumerate}
  \end{description}
  \begin{rem} Системы аксиом $S_1$ и $S_2$ равнозначны (эквивалентны,
т.\,е. порождают одно и то же множество ППФ. Выбор той или иной
системы определяется конкретными задачами. Для $S_1$ выводы будут
короче, но она более богатая; $S_2$ "--- более обозримая, более
компактная.
  \end{rem}
\item[$R$]:
  \begin{enumerate}
  \item \emph{Правило подстановки}. Если $\mathcal A$ "--- некоторая
выводимая формула $\mathcal A (A)$, то выводима и форма, получающаяся
из первой заменой всех вхождений $A$ на формулу $\mathcal B$:
$\mathcal{A(B)}$. Обозначение:
    $$ \frac{\mathcal A (A)}{\mathcal A (\mathcal B)}\quad или\quad
    \mathcal A (A) \vdash \mathcal A (\mathcal B)\,.
    $$
  \item \emph{Modus Ponens} (\emph{MP}) "--- правило заключения. Если
$\mathcal A$ и $\mathcal{A\supset B}$ "--- выводимые формулы, то
выводима и формула $B$. Обозначение:
    $$ \mathcal{\frac{A,\;A\supset B}B}\quad или\quad
    \mathcal{A,\;A\supset B\vdash B}\,.
    $$
  \item \emph{Теорема дедукции.} Пусть некоторая $\Gamma$ "---
множество формул, $A$, $B$ "--- тоже формулы, тогда если
    \begin{gather*} \Gamma,\mathcal A\vdash \mathcal B \Rightarrow
\Gamma \vdash \mathcal A \supset \mathcal B\,,\\ \frac{\mathcal
A}{\mathcal{A\supset B}}\,,\quad \text{где}\ \mathcal B\ \text{---
любая ППФ}\,.
    \end{gather*}
  \end{enumerate}
  \begin{rem} Система аксиом $S$ и правила $R$ (вместе с $U$ и $L$)
полностью определяют множество всех ППФ исчисления высказываний.
  \end{rem}

  \begin{ex}[вывода в исчислении высказываний (ИВ)] Покажем
выводимость формулы, воспользовавшись $S_{II}$:
    \begin{enumerate}
    \item $\frac{II.2\;(B)}{II.2\;(A\supset A)}$,
$\frac{II.2\;(C)}{II.2\;(A)}$:

      $\vdash\Bigl(A\supset\bigl(\left(A\supset A\right)\supset
A\bigr)\Bigr)\supset\Bigl(\bigl(A\supset\left(A\supset
A\right)\bigr)\supset\left(A\supset A\right)\Bigr)$.
    \item $\frac{II.1\;(B)}{II.1\;(A\supset A)}$:\quad $\vdash
A\supset\bigl(\left(A\supset A\right)\supset A\bigr)$.
    \item MP 1, 2:\quad $\vdash\bigl(A\supset\left(A\supset
A\right)\bigr)\supset\left(A\supset A\right)$.
    \item $\frac{II.1\;(B)}{II.1\;(A)}$:\quad $\vdash
A\supset(A\supset A)$.
    \item MP 3, 4:\quad $\vdash(A\supset A)$.
    \end{enumerate}
  \end{ex}
\end{description}

%-------------------------------------------------------------------------------
\subsection{Логические функции. Определение логической функции}

Рассмотрим некоторое множество $B_2\{0,1\}$, а также различные
операции, которые определены на этом множестве. Таким образом можно
сформулировать алгебру логических функций.

Алгебра, образованная множеством $B_2$ вместе со всеми возможными
операциями на нём называется \emph{алгеброй логики}.

Функцией алгебры логики от $n$ переменных называется $n$-арная
операция на множестве $B_2$.
$$ \left\{
\begin{aligned} f^{(n)}(x_1,x_2,\ldots,x_n)\in\{0,1\},\\ \forall
x_1,x_2,\ldots,x_n \in\{0,1\}.
\end{aligned} \right.
$$
$$ f^{(n)}\colon \underbrace{B_2 \times B_2 \times \ldots \times
  B_2}_{\text{n раз}}\rightarrow B_2\quad\Rightarrow\quad
f^{(n)}\colon B_2^n\rightarrow B_2 \,.
$$

Все функции $f^{(n)}$ объединяются множеством $P_2=\{f^{(n)}\}\,$.

Алгебра, образованная $k$-элементным множеством $B_k\,,\;k=1,2,\ldots$
вместе со всеми операциями на нём, называется \emph{алгеброй
$k$-значной логики}. $n$-арные операции на $k$-элементном множестве
$B_k$, называются $k$-значными логическими функциями. Множество всех
$k$-значных логических функций $\{f_k^{(n)}\}=P_k\,$.

%-------------------------------------------------------------------------------
\subsubsection{Логические функции одной переменной}

$n=1$.

\begin{center}
  \begin{tabular}{c || c | c | c | c |} $x$ & $\varphi_0$ &
$\varphi_1$ & $\varphi_2$ & $\varphi_3$\\ \hline\hline 0 & 0 & 0 & 1 &
1\\ \hline 1 & 0 & 1 & 0 & 1\\ \hline
  \end{tabular}
\end{center}

$\varphi_0(x)=0\,$,\quad $\varphi_1(x)=x\,$,\quad
$\varphi_2(x)=\overline\varphi_2\,$,\quad $\varphi_3(x)=1\,$.

\subsubsection{Логические функции 2-х переменных} $n=2$

\begin{center} \tabcolsep=5dd
  \begin{tabular}{c|c||c|c|c|c|c|c|c|c|c|c|c|c|c|c|c|c|}
$x_1$&$x_2$&$\varphi_0$&$\varphi_1$&$\varphi_2$&$\varphi_3$&$\varphi_4$&$\varphi_5$&$\varphi_6$&$\varphi_7$&$\varphi_8$&$\varphi_9$&$\varphi_{10}$&$\varphi_{11}$&$\varphi_{12}$&$\varphi_{13}$&$\varphi_{14}$&$\varphi_{15}$\\
\hline\hline 0&0&0&0&0&0&0&0&0&0&1&1&1&1&1&1&1&1\\ \hline
0&1&0&0&0&0&1&1&1&1&0&0&0&0&1&1&1&1\\ \hline
1&0&0&0&1&1&0&0&1&1&0&0&1&1&0&0&1&1\\ \hline
1&1&0&1&0&1&0&1&0&1&0&1&0&1&0&1&0&1\\ \hline
  \end{tabular}
\end{center}

\begin{center}
  \begin{align*} \varphi_0(x_1,x_2)&=0\,,&
\varphi_1(x_1,x_2)&=x_1\with x_2\,,&
\varphi_2(x_1,x_2)&=\overline{x_1\to x_2}\,,\\
\varphi_3(x_1,x_2)&=x_1\,,&
\varphi_4(x_1,x_2)&=\overline{x_1\leftarrow x_2}\,,&
\varphi_5(x_1,x_2)&=x_2\,,\\ \varphi_6(x_1,x_2)&=x_1\oplus x_2\,,&
\varphi_7(x_1,x_2)&=x_1\lor x_2\,,&
\varphi_8(x_1,x_2)&=x_1\;\downarrow\;x_2\,,&\\
\varphi_9(x_1,x_2)&=x_1\sim x_2\,,& \varphi_{10}(x_1,x_2)&=\overline
x_2\,,& \varphi_{11}(x_1,x_2)&=x_1\leftarrow x_2\,,\\
\varphi_{12}(x_1,x_2)&=\overline x_1\,,&
\varphi_{13}(x_1,x_2)&=x_1\rightarrow x_2\,,&
\varphi_{14}(x_1,x_2)&=x_1\mid x_2\,,\\ \varphi_{15}(x_1,x_2)&=1\,.
  \end{align*}
\end{center}

%-------------------------------------------------------------------------------
\subsection{Разложение логических функций по переменным}

$$
\left.
\begin{aligned} &x^0=\overline x\,,\\ &x^1=x\,,\\
&\alpha\in{0,1}=B_2\,.
\end{aligned} \right\} \Rightarrow
\begin{aligned} x^\alpha&=1,& \text{если $x=\alpha$,}\\ x^\alpha&=0,&
\text{если $x\ne\alpha$.}
\end{aligned}
$$

\begin{theorem}[о разложении функций по переменным] Всякая логическая
функция $f(x_1,x_2,\ldots,x_n)$ может быть представлена в виде:

  \begin{equation}
    \label{logicfunc} \bigvee_{\left<
\alpha_1,\alpha_2,\ldots,\alpha_m \right> \subseteq
B_2^m}x_1^{\alpha_1},x_2^{\alpha_2},\ldots,x_m^{\alpha_m}
f(\alpha_1,\alpha_2,\ldots,\alpha_m,x_{m+1},\ldots,x_n)\,,
  \end{equation}
  
  где $m\leqslant n$, и дизъюнкция берется по всем $2^m$ наборам
$\left< \alpha_1,\alpha_2,\ldots,\alpha_m \right>$ значений переменных
$x_1,x_2,\ldots,x_n$.
\end{theorem}

\begin{ex} Разложим логическую функцию $f(x_1,x_2,x_3,x_4)$ по
переменным $x_1,x_2$, \linebreak[4] $m=2$\,:
  \begin{multline*} f(x_1,x_2,x_3,x_4) = \\ =
x_1^0\cdot{}x_2^0\cdot{}f(0,0,x_3,x_4) \lor
x_1^0\cdot{}x_2^1\cdot{}f(0,1,x_3,x_4) \lor
x_1^1\cdot{}x_2^0\cdot{}f(1,0,x_3,x_4) \lor
x_1^1\cdot{}x_2^1\cdot{}f(1,1,x_3,x_4) = \\ =
\overline{x_1}\cdot{}\overline{x_2}\cdot{}f(0,0,x_3,x_4) \lor
\overline{x_1}\cdot{}x_2\cdot{}f(0,1,x_3,x_4) \lor
x_1\cdot{}\overline{x_2}\cdot{}f(1,0,x_3,x_4) \lor
x_1\cdot{}x_2\cdot{}f(1,1,x_3,x_4)
  \end{multline*}
\end{ex}

\begin{proof} Подставим в обе части равенства~\eqref{logicfunc}
произвольный набор \linebreak $\left<
\sigma_1,\sigma_2,\ldots,\sigma_n
\right>$\,. Т.\,к. $x_i^{\alpha_i}=1$, то среди $2^m$ конъюнкций
$x_1^{\alpha_1}x_2^{\alpha_2}\ldots{}x_m^{\alpha_m}$ в единицу
обратится только одна, причём та, в которой
$x_1^{\alpha_1}=x_2^{\alpha_2}=\ldots=x_m^{\alpha_m}=1$\,, т.\,е. в
которой $x_1=\alpha_1$, $x_2=\alpha_2$, \ldots, $x_m=\alpha_m$\,. Это
означает, что:
  $$
  f(\sigma_1, \sigma_2, \ldots, \sigma_n) = \sigma_1^{\alpha_1}
\sigma_2^{\alpha_2} \ldots \sigma_m^{\alpha_m} f(\sigma_1, \sigma_2,
\ldots, \sigma_m, x_{m+1}, \ldots, x_n) \,.
  $$
\end{proof}

\begin{conseq}[СДНФ] Разложение логической функции по всем переменным
и формирование СДНФ:

  \begin{equation} f(x_1,x_2,\ldots,x_n)=
\bigvee_{f(\sigma_1,\sigma_2,\ldots,\sigma_n)=1}
x_1^{\sigma_1}x_2^{\sigma_2},\ldots,x_n^{\sigma_n}
  \end{equation}
\end{conseq}

\begin{ex} $f(x_1, x_2, x_3)$\,:\\ \\
  \begin{tabular}{c|c|c||c} $x_1$ & $x_2$ & $x_3$ & $f$\\ \hline
\hline 0 & 0 & 0 & 0\\ \hline 0 & 0 & 1 & 1\\ \hline 0 & 1 & 0 & 0\\
\hline 0 & 1 & 1 & 0\\ \hline 1 & 0 & 0 & 1\\ \hline 1 & 0 & 1 & 1\\
\hline 1 & 1 & 0 & 0\\ \hline 1 & 1 & 1 & 0\\
  \end{tabular} \qquad $ f(x_1, x_2, x_3) =
\overline{x_1}\,\overline{x_2}\,x_3 \lor
x_1\,\overline{x_2}\,\overline{x_3} \lor x_1\,\overline{x_2}\,x_3\,. $
\end{ex}

\begin{conseq}[СКНФ] Разложение логической функции по всем переменным
и формирование СКНФ:

  \begin{equation} f(x_1,x_2,\ldots,x_n)=
\bigwith_{f(\sigma_1,\sigma_2,\ldots,\sigma_n)=0}
x_1^{\sigma_1}x_2^{\sigma_2},\ldots,x_n^{\sigma_n}
  \end{equation}
\end{conseq}

%-------------------------------------------------------------------------------

\subsection{Булева алгебра. Свойства булевой алгебры}

\begin{defin}[булевой алгебры]
  $$ 
  A = (P_2;\lor,\with,\lnot)
  $$
  \begin{enumerate}
  \item Ассоциативность относительно:
    \begin{itemize}
    \item конъюнкции: $x_1 \with (x_2 \with x_3) = (x_1 \with x_2)
\with x_3$\,;
    \item дизъюнкции: $x_1 \lor (x_2 \lor x_3) = (x_1 \lor x_2) \lor
x_3$\,.
    \end{itemize}
  \item Коммутативность относительно:
    \begin{itemize}
    \item конъюнкции: $x_1 \with x_2 = x_2 \with x_1$\,;
    \item дизъюнкции: $x_1 \lor x_2 = x_2 \lor x_1$\,.
    \end{itemize}
  \item Дистрибутивность:
    \begin{itemize}
    \item конъюнкции относительно дизъюнкции: $x_1 \with (x_2 \lor
x_3) = x_1 \with x_2 \lor x_1 \with x_3$\,;
    \item дизъюнкции относительно конъюнкции: $x_1 \lor (x_2 \with
x_3) = (x_1 \lor x_2) \with (x_1 \lor x_3)$\,.
    \end{itemize}
  \item Идемпотентность:
    \begin{itemize}
    \item конъюнкции: $x \with x = x$\,;
    \item дизъюнкции: $x \lor x = x$\,.
    \end{itemize}
  \item $\lnot\lnot x = x$\,.
  \item Свойство констант:
    \begin{itemize}
    \item $x \with 1 = x$\,;
    \item $x \with 0 = 0$\,;
    \item $x \lor 1 = 1$\,;
    \item $x \lor 0 = x$\,;
    \item $\lnot 0 = 1$\,.
    \end{itemize}
  \item Правила де Моргана:
    \begin{itemize}
    \item $\overline{x_1 \with x_2} = \overline{x_1} \lor
\overline{x_2}$\,;
    \item $\overline{x_1 \lor x_2} = \overline{x_1} \with
\overline{x_2}$\,.
    \end{itemize}
  \item Закон противоречия: $x \with \overline{x} = 0$\,.
  \item Закон исключения 3-его: $x \lor \overline{x} = 1$\,.
  \end{enumerate}
\end{defin}

%-------------------------------------------------------------------------------
\subsection{Эквивалентные преобразования в булевой алгебре}
\begin{enumerate}
\item Свойство поглощения:
  \begin{itemize}
  \item $x \lor xy = x$\,;
  \item $x \with (x \lor y) = x$\,.
  \end{itemize}
\item Свойство склеивания и расщепления: $xy \lor x\overline{y} = x$\,.
\item Обобщённое склеивание: $xz \lor y\overline{z} \lor xy = xz \lor
y\overline{z}$.
\item $x \lor \overline{x}y = x \lor y$\,.
\item Обобщение: $x_1 \lor f(x_1,x_2,\ldots,x_n) = x_1 \lor
f(0,x_2,\ldots,x_n)$\,.
\end{enumerate}

%-------------------------------------------------------------------------------
\subsection{Интерпретация формул исчисления высказываний}

Как уже рассматривалось ранее, в исчислении высказываний каждой
формуле должно быть приписано истинностное значение или элемент
множества:
$$ \left\{ И,Л \right\} = \left\{ T,F \right\} = \left\{ 1,0 \right\}
\,.$$

Приписывая каждому атомарному символу из формулы $F$ истинностные
значения, мы будем получать и различные истинностные значения и самой
формулы $F$\,.

Таблицу, в которой указаны истинностные значения $F$ при всевозможных
истинностных значениях входящих атомов, назовём \emph{таблицей
истинности}.

\begin{ex} $F = (A \lor B) \supset C$
  \begin{center}
    \begin{tabular}{c|c|c||c|c|} $A$ & $B$ & $C$ & $A \lor B$ & $F$\\
\hline\hline 0 & 0 & 0 & 0 & 1\\ \hline 0 & 0 & 1 & 0 & 1\\ \hline 0 &
1 & 0 & 1 & 0\\ \hline 0 & 1 & 1 & 1 & 1\\ \hline 1 & 0 & 0 & 1 & 0\\
\hline 1 & 0 & 1 & 1 & 1\\ \hline 1 & 1 & 0 & 1 & 0\\ \hline 1 & 1 & 1
& 1 & 1\\ \hline
    \end{tabular}
  \end{center}
\end{ex}

\begin{defin}[интерпретации] Пусть $F$ "--- некоторая формула
исчисления высказываний, $A_1,A_2,\ldots,A_n$ "--- её атомарные
символы. \emph{Интерпретацией} $F$ является такое приписывание
истинностных значений атомам $A_1,A_2,\ldots,A_n$, при котором каждому
$A_i$ приписывается либо 1 либо 0.
\end{defin}

Говорят, что $F$ истинна при некоторой интерпретации тогда и только
тогда, когда $F$ получает значение 1 в этой интерпретации; в противном
случае $F$ ложна при этой интерпретации.

%-------------------------------------------------------------------------------
\subsection{Логические следствия в исчислении высказываний}

В математической логике следует различать понятия выводимости и
логического следования.

\begin{ex}[логическое следование] Событие $A$ "--- студент пришёл на
экзамен. $B$ "--- вытащил билет, ответ на который он знает. $C$ "---
экзамен сдан успешно.
  \begin{center} $A \with B \Rightarrow C$,\qquad $C$ "--- логическое
следование.
  \end{center}
\end{ex}

\begin{defin} Пусть даны формулы: $F_1,F_2,\ldots,F_n,G$. Говорят, что
$G$ "--- логическое следствие $F_1,F_2,\ldots,F_n$ или $G$ следует из
$F_1,F_2,\ldots,F_n$ тогда и только тогда, когда для всякой
интерпретации $I$, в которой $I\colon
F_1\with{}F_2\with\ldots\with{}F_n = 1$,\quad$G = 1$. В этом случае
говорят, что $F_1,F_2,\ldots,F_n$ "--- посылки, а $G$ "--- следствие, и
обозначают этот факт
  \begin{equation} F_1,F_2,\ldots,F_n \models G\,.
  \end{equation}
\end{defin}

\begin{theorem} Пусть даны $F_1,F_2,\ldots,F_n$ и $G$, тогда $G$ есть
логическое следствие формул $F_1,F_2,\ldots,F_n$ тогда и только тогда,
когда формула $(F_1 \with F_2 \with\ldots\with F_n) \supset G$
является общезначимой.
\end{theorem}

\begin{theorem} Пусть даны $F_1,F_2,\ldots,F_n$ и $G$. Формула $G$
есть логическое следствие формул $F_1,F_2,\ldots,F_n$ тогда и только
тогда, когда $F_1 \with F_2 \with\ldots F_n \with \overline{G}$
противоречива.
\end{theorem}

\begin{ex} $F_1\colon P \supset Q$,\quad $F_2\colon
\overline{Q}$,\quad $G\colon \overline{P}$,\quad где $P$ и $Q$ "---
атомы. Доказать, $F_1,F_2 \models G$.
  \begin{center}
    \begin{tabular}{c|c||c|c|c|c|c|c|} $P$ & $Q$ & $P \supset Q$ &
$\overline{Q}$ & $F_1 \with F_2$ & $\overline{P}$ & $F_1 \with F_2
\supset G$ & $F_1 \with F_2 \with \overline{G}$\\ \hline\hline 0 & 0 &
1 & 1 & 1 & 1 & 1 & 0\\ \hline 0 & 1 & 1 & 0 & 0 & 1 & 1 & 0\\ \hline
1 & 0 & 0 & 1 & 0 & 0 & 1 & 0\\ \hline 1 & 1 & 1 & 0 & 0 & 0 & 1 & 0\\
\hline
    \end{tabular}
  \end{center}
\end{ex}

%===============================================================================
\section{Исчисление предикатов или теории первого порядка}

\subsection{Определение исчисления предикатов}

\emph{Предикатом} $P(x_1,x_2,\ldots,x_n)$ называется функция $P$,
ставящая в соответствие $M^m\rightarrow B$ или $P\colon \underbrace{M
\times M \times\ldots\times M}_{\text{m раз}} \rightarrow B$, где
$M$ "--- произвольное множество , в том числе и бесконечное, называемое
предметной областью, $B = \{0,1\}$ "--- булево множество,
$x_1,x_2,\ldots,x_m$ "--- предметные переменные.

Другими словами, $m$-местный предикат, определенный на множестве
$M$ "--- это двухзначная функция от $m$ аргументов, принимающих
значение в произвольном множестве $M$.

Таким образом, видно, что предикат отличается от высказывания наличием
предметных переменных, которые могут принимать значения из некоторой
предметной области. Другим отличием предикатов является наличие
кванторов.

\begin{ex} $\exists x P(x)$,\quad $\forall x \overline{P}(x)$\,.
\end{ex}

\begin{defin} Переход от предиката P(x)
  
  $$
  P(x) \rightarrow
  \begin{cases} \exists x P(x)\\ \forall x P(x)
  \end{cases}
  $$
  называется операцией \emph{связывания переменной}, или операцией
\emph{квантификации}, а переменная $x$ "--- связанной. Если переменная
в предикате не связана, то она свободная.
\end{defin}

\begin{defin} Выражение или формула, на которые распространяется
действие квантора, называется \emph{областью действия квантора}.
\end{defin}

$$FT = \left<U,L,S,R\right>$$

\begin{description}
  \item[Алфавит] исчисления предикатов включает:
    \begin{itemize}
    \item $X = \{x_i\}$ "--- множество предметных переменных $x_i \in
M$.
    \item $A = \{a_j\}$ "--- множество предметных констант $a_j \in M$.
    \item $P = \{P_1^1,P_2^1,\ldots,P_k^l,\ldots\}$ "--- множество
предикатных букв, где $l$ "--- арность (местность) $k$-го предиката.
    \item $F = \{f_1^1,f_2^1,\ldots,f_k^l,\ldots\}$ "--- множество
функциональных букв, где $l$ "--- арность (местность) $k$-ой функции.
    \item знаки логических связок: $\lor$, $\with$, $\lnot$,
$\supset$, \ldots.
    \item кванторы: $\exists$, $\forall$, \ldots.
    \item служебные: .\quad,\quad(\quad)\quad \ldots.
    \end{itemize}
  \item[Язык] определяется в 2 этапа:
    \begin{enumerate}
    \item Термы (элементы $M$), $M \rightarrow B$:
      \begin{itemize}
      \item $x_i \in X$, $a_j \in A$;
      \item если $f^n \in F$ "--- функциональная буква,
$t_1,t_2,\ldots,t_n$ "--- термы, то термом также является
$f^n(t_1,t_2,\ldots,t_n)$.
      \end{itemize}
    \item Формулы:
      \begin{itemize}
      \item если $p^l \in P$, а $t_1,t_2,\ldots,t_l$ "--- термы, то
$P^l(t_1,t_2,\ldots,t_l)$ "--- формула;
      \item если $F_1$, $F_2$ "--- формулы, то формулами также являются
$F_1 \lor F_2$, $\overline{F_1}$, $\overline{F_2}$,~\ldots.
        \begin{rem} Все переменные, входящие в таким образом
построенные формулы, являются \emph{свободными}.
        \end{rem}
      \item если $F(x)$ "--- некоторая формула, в которой $x$ "---
свободная переменная, то $\exists x F(x)$ и $\forall x F(x)$ "--- также
формулы.
        \begin{rem} Формулы, построенные таким образом, являются
связанными.
        \end{rem}
      \end{itemize}
    \end{enumerate}
  \item[Аксиомы] исчисления предикатов делятся на 2 группы:
    \begin{enumerate}
    \item $S_I$ и $S_{II}$ "--- аксиомы исчисления высказываний;
    \item предикатные аксиомы:
      \begin{enumerate}
      \item $\forall x F(x) \supset F(y)$;
      \item $F(y) \supset \exists x F(x)$.
      \end{enumerate}
    \end{enumerate}
  \item[Правила вывода] исчисления предикатов содержат:
    \begin{enumerate}
    \item Modus Ponens;
    \item Правило обобщения (введения квантора $\forall$): $\frac{F
\supset G(x)}{F \supset \forall x G(x)}$
    \item Правило введения $\exists$: $\frac{G(x) \supset F}{\exists x
G(x) \supset F}$
    \end{enumerate}
\end{description}

%-------------------------------------------------------------------------------
\subsection{Интерпретация формул исчисления предикатов}

Интерпретация формулы F в исчислении предикатов определяется
сопоставлением:
\begin{enumerate}
\item Каждой предметной константе некоторого элемента из $M$.
\item Каждому $n$-местному функциональному символу отображения
$f\colon M^n \rightarrow M$.
\item Каждому $m$-местному предикатному символу $P\colon M^m
\rightarrow B$.
\end{enumerate}

\begin{ex} $F\colon \forall x P(x) = Л$.

  $M = \{1,2\}$.

  \begin{center}
    \begin{tabular}{c||c|c|} $M$ & 1 & 2\\ \hline\hline $P(x)$ & И(1)
& Л(0)\\
    \end{tabular}
  \end{center}

  Формула $F$ ложна в данной интерпретации.
\end{ex}

\begin{ex} $F\colon \forall x (P(x) \supset Q(f(x),a))$. Определить
интерпретацию и определить истинность в данной интерпретации.

  $M = \{1,2\}$,\qquad $a = 1$,\quad $a \in M$.

  \begin{center}
    \begin{tabular}{c||c|c|} $M$ & 1 & 2\\ \hline\hline $f(n)$ & 2 &
1\\
    \end{tabular} \qquad
    \begin{tabular}{c||c|c|} $M$ & 1 & 2\\ \hline\hline $P(x)$ & Л(0)
& И(1)\\
    \end{tabular}
  \end{center}
  
  \begin{center}
    \begin{tabular}{c||c|} $(x,y)$ & $Q(x,y)$\\ \hline\hline $(1,1)$ &
И(1)\\ \hline $(1,2)$ & И(1)\\ \hline $(2,1)$ & Л(0)\\ \hline $(2,2)$
& И(1)\\ \hline
    \end{tabular}
  \end{center}
\end{ex}

%-------------------------------------------------------------------------------

\subsection{Эквивалентные преобразования в исчислении предикатов}

\begin{defin} Формулы называются эквивалентными, если при любых
интерпретациях они принимают одно значение.
\end{defin}

\begin{enumerate}
\item $\overline{\exists x P(x)} = \forall x \overline{P(x)}$;
\item $\overline{\forall x P(x)} = \exists x \overline{P(x)}$;

\item[] Дистрибутивность операции $\forall$ относительно конъюнкции:
\item $\forall x (P_1(x) \with P_2(x)) = \forall x P_1(x) \with
\forall x P_2(x)$;

\item[] Дистрибутивность квантора существования относительно
дизъюнкции:
\item $\exists x (P_1(x) \lor P_2(x)) = \exists x P_1(x) \lor \exists
x P_2(x)$;

\item[] Односторонние:
\item $\exists x (P_1(x) \with P_2(x)) \Rightarrow \exists x P_1(x)
\with \exists x P_2(x)$;
\item $\forall x P_1(x) \lor \forall x P_2(x) \Rightarrow \forall x
(P_1(x) \lor P_2(x))$;

\item[] Коммутативность одинаковых кванторов:
\item $\forall x \forall y P(x,y) = \forall y \forall x P(x,y)$;
\item $\exists x \exists y P(x,y) = \exists y \exists x P(x,y)$;

\item[] Пусть $Y$ "--- некоторая формула, не содержащая переменную $x$:
\item $\forall x (P(x) \with Y) = \forall x P(x) \with Y$;
\item $\forall x (P(x) \lor Y) = \forall x P(x) \lor Y$;
\item $\exists x (P(x) \with Y) = \exists P(x) \with Y$;
\item $\exists x (P(x) \lor Y) = \exists x P(x) \lor Y$;
\end{enumerate}

\begin{rem} Поскольку исчисление предикатов есть расширение исчисления
высказываний, то для формул исчисления предикатов остаются верными и
эквивалентные предикаты, действующие для исчисления высказываний.
\end{rem}

%-------------------------------------------------------------------------------
\subsection{Нормальная форма в исчислении предикатов}

В исчислении предикатов так же, как и в исчислении высказываний,
имеются \emph{нормальные формы} (НФ), в частности, ПНФ "--- пренексная
нормальная форма:
\begin{equation} (Q_1x_1Q_2x_2 \ldots Q_nx_n)C\,,
\end{equation} где $Q_1,Q_2,\ldots,Q_n \in \{\exists,\forall\}$,
$(Q_1x_1,Q_2x_2,\ldots,Q_nx_n)$ "--- префикс, $C$ "--- матрица-формула в
КНФ (бескванторное логическое выражение в КНФ).

Любая формула исчисления предикатов может быть преобразована в ПНФ с
помощью некоторого алгоритма. Рассмотрим работу этого алгоритма на
примере:
\begin{ex} $F\colon \forall{x}(P_1(x)) \supset \lnot\forall{x}(P_2(y)
\lor \exists{y}P_3(x,y))$. Задача: построить ПНФ.
  \begin{description}
    \item[1 шаг:] поскольку КНФ предполагает использование логических
связок $\lor$, $\with$, $\lnot$, то нужно избавиться от $\supset$:\\
$A \supset B = \overline{A} \lor B$\,;\\
$\lnot\forall{x}\bigl(P_1(x)\bigr) \lor \lnot\forall{x}\bigl(P_2(y)
\lor \exists{y}P_3(x,y)\bigr)$\,.
    \item[2 шаг:] изменить (уменьшить) область действия $\lnot$:\\
$\exists{x}\bigl(\overline{P_1}(x)\bigr) \lor
\exists{x}\overline{\bigl(P_2(y)\lor\exists{y}P_3(x,y)\bigr)} =
\exists{x}\overline{P_1}(x) \lor
\exists{x}\bigl(\overline{P_2}(y)\with\lnot\exists{y}P_3(x,y)\bigr)
=$\\ $=\exists{x}\overline{P_1}(x) \lor \exists{x}\overline{P_2}(y)
\with \forall{y}\overline{P_3}(x,y)$\,.
    \item[3 шаг:] Переобозначим необходимые переменные для того, чтобы
не было коллизий (или не было одинаковых):\\
$\exists{u}\overline{P_1}(u) \lor \exists{x}\overline{P_2}(z) \with
\forall{y}\overline{P_3}(x,y)$\,.
    \item[4 шаг:] Формирование префикса для ПНФ:\\
$\exists{u}\overline{P_1}(u) \lor
\exists{x}\forall{y}\bigl(\overline{P_2}(z) \with
\overline{P_3}(x,y)\bigr) = \exists{u}\Bigl(\overline{P_1}(u) \lor
\exists{x}\forall{y}\bigl(\overline{P_2}(z) \with
\overline{P_3}(x,y)\bigr)\Bigr)=$\\ $=\exists{u}\exists{x}
\Bigl(\overline{P_1}(u) \lor \forall{y}\bigl(\overline{P_2}(z) \with
\overline{P_3}(x,y)\bigr)\Bigr) = \exists{u}\exists{x}\forall{y}
\bigl(\overline{P_1}(u) \lor \overline{P_2}(z) \with
\overline{P_3}(x,y)\bigr)$\,.
    \item[5 шаг:] Построение матрицы $C$ (КНФ):\\
$\exists{u}\exists{x}\forall{y}\Bigl(\bigl(\overline{P_1}(u) \lor
\overline{P_2}(z)\bigr) \with \bigl(\overline{P_1}(u) \lor
\overline{P_3}(x,y)\bigr)\Bigr)$\,.
  \end{description}
\end{ex}

%-------------------------------------------------------------------------------

\subsection{Доказательство теорем в исчислении предикатов}

Доказательство формул исчисления предикатов путём подстановки в них
констант называется \emph{методом интерпретации}. Этот метод позволяет
интерпретировать формулу как осмысленное утверждение, в связи с чем
такой метод называется \emph{семантическим}. Этот метод применим в
исчислении предикатов, вогда предметная область $M$ конечна. В таком
случае $M$ содержит конечное число элементов $M =
\{a_1,a_2,\ldots,a_n\}$.

В соответствии с семаническим подходом (методом), формула вида
$\forall{x}P(x)$ преобразуется в следующую ($x \in M$):
\begin{gather*} \forall{x}P(x) = P(a_1) \with P(a_2) \with \ldots
\with P(a_m)\\ \exists{x}P(x) = P(a_1) \lor P(a_2) \lor \ldots \lor
P(a_m)
\end{gather*}

Истинность таким образом преобразованных формул может быть проверена
путем конечного количества подстановок и вычислений. В случае же,
когда предметная область $M$ бесконечна ($|M| = \infty$),
семантический метод неприемлем. В этом случае используют процедуры
логического вывода с применением аксиом и правил вывода. Такой подход
называется \emph{формальным}.

%===============================================================================

\section{Метод резолюций и его использование в системах искусственного
интеллекта}

В этом разделе будут рассмотрены процедуры (алгоритмы) поиска
доказательств, которые могут быть использованы при проектировании
прикладных систем искусственного интеллекта.

%-------------------------------------------------------------------------------

\subsection{История поиска процедур доказательства теорем}

Проблема: разработка общего алгоритма для проверки общезначимости
формул исчисления высказываний и исчисления предикатов.

\begin{description}
  \item[1646--1716~гг.] Лейбниц, первые алгоритмы.
  \item[XVI век.] Пеан\'{о}.
  \item[1920-е гг.] Д.\,Гильберт.
  \item[1930-е гг.] А.\,Чёрч, А.\,Тьюринг независимо доказали, что не
существует общей решающей процедуры (алгоритма), проверяющей
общезначимость формул в исчислении предикатов. Тем не менее,
существуют алгоритмы поиска доказательств, которые могут подтвердить
общезначимость формулы, если она общезначима.
  \item[1930 г.] Эрбран Жак разработал алгоритм нахождения
интерпретации, которая опровергает заданную формулу, однако, если
формула общезначима, то такой интерпретации не существует, и алгоритм
заканчивает работу за конечное число ходов. Этот подход лёг в основу
современных методов.
  \item[1960 г.] Гилмор "--- попытка реализации метода Эрбрана на ЭВМ,
исходя из тезиса: <<формула общезначима тогда и только тогда, когда её
отрицание противоречиво>>. Программа Гилмора обнаруживала
противоречивость данной формулы, а значит, подтверждала её
общезначимость.
  \item[1960-е гг.] Если отрицание формулы противоречиво, то программа
обнаруживала этот факт, но была очень громоздкой. Дэвис и Патнем
доказали чрезмерную громоздкость и неприменимость в практике программы
Гилмора.
  \item[1965 г.] Робинсон Джон предложил наиболее эффективный алгоритм
поиска доказательства, который оказался самым эффективным. Сейчас
применяются много оптимизированных для разных областей вариаций этого
алгоритма:
    \begin{itemize}
    \item семантическая резолюция;
    \item лок-резолюция;
    \item линейная резолюция;
    \item алгоритм британского музея;
    \item и др.
    \end{itemize}
\end{description}

%-------------------------------------------------------------------------------

\subsection{Метод резолюций для исчисления высказываний}

\begin{theorem} Множество дизъюнктов $S=\{D_1,D_2,\dots,D_n\}$ "--- это
эквивалентная формулировка исходной теоремы:
  $$F = (F_1 \with F_2 \with \ldots \with F_n \supset \overline{G})\,,$$
  у которой $F_1,F_2,\ldots,F_n$ "--- посылки, $G$ "--- заключение, а
доказательством этой теоремы является опровержение (нахождение
опровержения) $F$.
\end{theorem}

\begin{proof} Доказательство теоремы сводится к доказательству
невыполнимости $S$. Если ввести в рассмотрение пустой дизъюнкт
$\square \equiv Л$, то сказанное выше интерпретируется следующим
образом: если среди элементов $S$ имеется хотя бы 1 пустой дизъюнкт,
то множество $S$ будет опровергаться, причём в любой интерпретации,
следовательно, $S$ невыполнимо.
\end{proof}

Если к невыполнимом множеству $S$ добавить некоторое подмножество $S'$
дизъюнктов таких, что $S'\colon S \supset S'$, то невыполнимость $S$ и
$S'$ будет сохранена.

Если среди логических следствий $S$ появится пустой дизъюнкт $S \lor
S'$, то это означает, что $S$ невыполнимо, а это в свою очередь
значит, что целью доказательства теоремы, сформулированной выше,
является получение среди логических следствий множества $S$ пустого
дизъюнкта.

\begin{defin}[резольвенты] Для любых 2-х дизъюнктов $C_1$ и $C_2$,
если существует литера $L_1$ в $C_1$, которая контрарна некоторой
литере $L_2$ в $C_2$, то, вычеркнув $L_1$ и $L_2$ из $C_1$ и $C_2$
соответственно, можно построить дизъюнкцию оставшихся
дизъюнктов. Построенный таким образом дизъюнкт называется
\emph{резольвентой} $C_1$ и $C_2$.
\end{defin}

\begin{ex} Рассмотрим $C_1=P\lor{}R$, $C_2=\overline{P}\lor{}Q$.
  $$ \left.
  \begin{aligned} C_1 &= P \lor R\\ C_2 &= \overline{P} \lor Q
  \end{aligned} \right\} \quad C = R \lor Q\text{ "--- резольвента.}
  $$
\end{ex}

Важным свойством резольвенты является то, что она является следствием
$C_1$ и $C_2$.

\begin{theorem} Пусть даны 2 дизъюнкта: $C_1$ и $C_2$. Тогда
резольвента $C$ дизъюнктов $C_1$ и $C_2$ есть логическое следование
исходных дизъюнктов, а именно:
  \begin{equation} (C_1 \with C_2) \supset C\,.
  \end{equation}
\end{theorem}

\begin{defin}[правило резолюций или правило вывода логических
следствий или правило резолютивного вывода] Пусть $S$ "--- некоторое
множество дизъюнктов. Резолютивный вывод $C$ из множества дизъюнктов
есть последовательность $C_1,C_2,\ldots,C_n$ дизъюнктов, каждый из
которых ($C_i$) либо принадлежит $S$, либо является резольвентой
дизъюнктов, предшествующих $C_i$.

  Вывод пустого дизъюнкта $\square$ из $S$ называется опровержением,
или доказательством невыполнимости $S$.
\end{defin}

\begin{ex} $S = \{\overline{P}\lor{}Q,\; \overline{Q},\; P\}$

  Резольвента $\overline{P}\lor{}Q$ и $\overline{Q}$ $\Rightarrow$
$\overline{P}$

  Резольвента $P$ и $\overline{P}$ $\Rightarrow$ $\square$

  Из $S \supset \square$ следует, что $S$ невыполнима.
\end{ex}

Таким образом, резольвента строится для 2-х дизъюнктов, содержащих
контрарную пару. Если для формулы исчисления высказываний такие пары
находятся (описываются) достаточно просто, то для формул исчисления
предикатов такая задача решается сложнее. Это связано с тем, что в
формулах исчисления предикатов необходимо ещё и совпадение литер
контрарной пары.

Т.\,к. задача резолютивного вывода "--- поиск опровержения исходной
формулы, то для этого можно использовать и константно-частные случаи
соответствующих дизъюнктов.

Поиск константно-частных случаев для дизъюнктов в формулах исчисления
предикатов, является предметом алгоритма унификации.

%%% Local Variables: 
%%% mode: latex
%%% TeX-master: "lections"
%%% End: 